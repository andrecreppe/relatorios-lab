\subsection{Determinação da densidade de um líquido utilizando o
Areômetro de Nicholson}

Para os nossos cálculos, utilizaremos os seguintes dados que foram disponibilizados no vídeo da aula, em conjunto com um valor da densidade da água disponibilizado pela apostila:

\begin{table}[H]
    \centering
    \begin{tabular}{ |M{5cm}||M{2cm}||M{2cm}||M{2cm}|  }
        \hline
        \textbf{O que foi medido} & \textbf{Valor} & \textbf{Incerteza} & \textbf{Unidade}\\
        \hline
        Massa Aerômetro + Pesos ($m_{ar} + m_t$)         & 135,96    & $\pm$ 0,01 & g\\
        Massa dos Pesos na água ($m_t$)                     & 34,30     & $\pm$ 0,01 & g\\
        Massa dos Pesos no líquido desconhecido ($m_t'$)    & 52,21     & $\pm$ 0,01 & g\\
        \hline
        Densidade da água a 25°C ($\rho _{agua}$)           & 1,000 & - & g/$cm^3$\\
        \hline
    \end{tabular}
    \caption{Dados coletados para o experimento de aferição da densidade}
\end{table}

Assim sendo, podemos começar os cálculos. Primeiro vamos determinar o volume do aerômetro:

\[ V_{ar} = \frac{m_{ar} + m_t}{\rho _{agua}} \]
\[ V_{ar} = \frac{135,96}{1,000} = 135,96 (cm^3) \]

\[ \delta V_{ar} = \frac{\delta m_{ar} + \delta m_t}{\rho _{agua}} \]
\[ \delta V_{ar} = \frac{0,01 + 0,01}{1} = 0,02 (cm^3) \]

Com esse dado em mãos, conseguiremos assim determinar a densidade do fluido desconhecido:

\[ \rho _x = \rho _{agua} + \frac{m_t - m_t'}{V_{ar}} \]
\[ \rho _x = 1,000 + \frac{34,30 - 52,21}{135,96} = 0,868270078 (g/cm^3) \]

\[ \delta \rho _x = \rho _{agua} \cdot \frac{(\delta m_t + \delta m_t') \cdot V_{ar} + \delta V_{ar} \cdot (m_t - m_t')}{V_{ar}^2} \]
\[ \delta \rho _x = 1,000 \cdot \frac{(0,01 + 0,01) \cdot 135,96 + 0,02 \cdot (34,30 - 52,21)}{135,96^2} \]
\[ \delta \rho _x = 0,00012772434 (g/cm^3) \]

Dessa forma, ajustando os algarismos significativos teremos que a densidade do fluido desconhecido é:

\[ \mathbf{\therefore \rho _x = 0,8683 \pm 0,0001 (g/cm^3)} \]

Comparando esse resultado com uma tabela online, podemos dizer que o líquido transparente com a densidade mais próxima do nosso resultado seria o \textit{benzeno} ($\rho _{benzeno} = 0.876 g/cm^3$).\\

Entretanto, como o vídeo deixou uma dica sobre o que estava dentro da proveta, uma garrafa de álcool de limpeza. Ou seja, era para termos encontrado um valor próximo da densidade específica do etanol ($\rho _{etanol} = 0.789 g/cm^3$), portanto o nosso valor encontrado \textbf{não foi compatível com o esperado}.\\

Provavelmente ocorreram alguns problemas com o experimento. Talvez o álcool tenha evaporado consideravelmente, aumentando a concentração de água nele, levando a um aumento da densidade. Ou ainda, seja uma característica do álcool 70, que possui 30\% de água, e essa quantidade leva naturalmente a densidade ser diferente da do álcool puro ($\rho _{etanol}$).
