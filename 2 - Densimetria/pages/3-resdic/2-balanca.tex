\subsection{Determinação do volume e da densidade de um sólido com
uma balança}

Conforme descrito na Metodologia, os primeiros dados registrados foram as massas do sistema recipiente-líquido ($m_{r+l}$) e do sistema recipiente-líquido com o sólido submerso  ($m'_{r+l}$), que estão registrados na Tabela abaixo com suas respectivas incertezas inerentes ao método de aferição utilizado:

\begin{table}[H]
    \centering
    \begin{tabular}{ |M{5cm}||M{2cm}||M{2cm}||M{2cm}|  }
        \hline
        \textbf{O que foi medido} & \textbf{Valor} & \textbf{Incerteza} & \textbf{Unidade}\\
        \hline
        Recipiente-Líquido ($m_{r+l}$)         & 83,70    & $\pm$ 0,01 & g\\
        Recipiente-Líquido com Sólido Submerso ($m'_{r+l}$)                     & 86,23     & $\pm$ 0,01 & g\\
        Densidade da água a 25°C ($\rho _{agua}$)           & 1,000 & - & g/$cm^3$\\
        \hline
    \end{tabular}
    \caption{Massas do sistema recipiente-líquido ($m_{r+l}$) e do sistema recipiente-líquido com o sólido submerso  ($m'_{r+l}$).}
\end{table}

Com estes valores em mãos, calculamos o volume do sólido por meio da fórmula já enunciada anteriormente:

\[ V_s = m'_{r+l} - m_{r+l} \]
\[ V_s = 86,23 - 83,70 \]
\[ V_s = 2,53 cm^3 \]

A partir dos valores das massas e de suas respectivas incertezas, podemos determinar a incerteza final do volume encontrado para o sólido. Dessa forma, temos que a incerteza associada ao valor pode ser encontrada pela expressão a seguir:

\[ \sigma_s = \sigma_m - \sigma_{m'} \]
\[ \sigma_s = 0,01 + 0,01 \]
\[ \sigma_s = 0,02 cm^3 \]

Portanto, este é o volume do sólido em questão. Agora, resta calcular a densidade e para isso é necessário conhecer a massa do sólido submerso. Após a leitura da balança, a massa do sólido é: $m_{solido}$ = (21,52 $\pm$ 0,01) g. Em seguida, calculamos a densidade do sólido e sua incerteza utilizando as fórmulas: 

\[\rho_{solido} = \frac{m_{solido}}{V_s} \]
\[\rho_{solido} = \frac{21,53}{2,53} \]
\[\rho_{solido} = 8,505928 g/cm^3 \]

\[ \sigma_s = \frac{m \cdot \delta V + V \cdot \delta m}{V^2} = \frac{21,52 \cdot 0,02 + 2,53 \cdot 0,01}{2,53^2} = \frac{0,4304 + 0,0253}{6,4009}\]
\[ \sigma_s = \frac{0,4557}{6,4009} = 0,071193\]

\[\therefore \rho_{solido} = (8,50 \pm 0,07) g/cm^3 \]

Após o cálculo, comparamos a densidade com o valor tabelado e chegamos à conclusão de que o sólido é feito de latão.

\begin{figure}[H]
    \centering
    \includegraphics[scale=0.4]{images/Gráfico Densidades 2.png}
    \text{Densidade $(g/cm^3)$}
    \caption{Gráfico de distribuição de valores da densidade experimental }
\end{figure}