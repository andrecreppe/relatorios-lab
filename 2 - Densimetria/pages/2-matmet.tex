\newpage
\section{Materiais e métodos}

Os experimentos realizados nessa prática foram embasados no Princípio de Arquimedes, o qual afirma que um fluido exerce uma força ascendente em um corpo que se encontra mergulhado no mesmo fluido, sendo que o módulo dessa força corresponde ao peso do volume do líquido deslocado.

Assim, foi montado um dispositivo dentro do laboratório para verificar tal princípio, de modo que um pequeno cilindro fosse acoplado na extremidade inferior de uma mola posicionada verticalmente. Logo, quando o cilindro foi solto, seu peso (força gravitacional) deformou a mola até tal força se igualar com a força elástica proporcionada em função do tempo, enquanto uma régua posicionada verticalmente próximo à mola viabilizou a marcação da deformação.

\begin{figure}[H]
    \centering
    \includegraphics[scale=0.8]{images/Experimento1.png}
    \caption{Esquema de Forças atuando em uma balança de tração.}
\end{figure}

Dois casos foram realizados. O primeiro foi com o cilindro, sem submersão deste em um líquido, foi solto, enquanto o segundo houve submersão do corpo cilíndrico em um recipiente com água após ser solto.

% =============== EXPERIMENTO 2 ===================== %

\subsection{Determinação do volume e da densidade de um sólido com
uma balança}

O segundo experimento consiste em determinar o Volume de um sólido a partir da medida do Empuxo sofrido pelo objeto quando mergulhado em um líquido de densidade conhecida - nesse caso o líquido é a água. Para isso, será utilizado um sistema com uma balança que sofre a ação de uma “Força Normal”(N), que pode ser representado conforme a Imagem a seguir:

\begin{figure}[H]
    \centering
    \includegraphics[scale=0.8]{images/Experimento2.png}
    \caption{Representação do experimento realizado em um sistema de balança que sofre a ação de uma força normal (N).}
\end{figure}

Com base na imagem, primeiramente determina-se a massa do recipiente com o líquido que será utilizado para submergir o corpo (a). Como o conjunto formado pelo recipiente e o líquido está em equilíbrio ($\sum$F=0), tem-se que a força normal (N), exercida pela superfície da balança perpendicularmente à parede do recipiente, é igual à força Peso (P) do sistema recipiente-líquido. Dessa forma, realizando as devidas manipulações matemáticas, tem-se que a massa do recipiente-líquido é igual a seguinte expressão:

\[ N = P \]
\[ N = m_{r+l} \cdot g \]
\[\therefore m_{r+l} = \frac{N}{g} \]

Entretanto, como está sendo utilizado uma balança digital, pode-se simplificar a obtenção da massa do sistema recipiente-líquido pela simples leitura do valor registrado na balança.

\begin{figure}[H]
    \centering
    \includegraphics[scale=0.8]{images/Experimento2.2.png}
    \caption{Leitura da balança após incluir um recipiente com água.}
\end{figure}

Após isso, mergulha-se o corpo cujo volume se deseja determinar, segurando-o por um fio e tomando os devidos cuidados para que ele fique totalmente submerso no líquido e não toque nas laterais ou no fundo do recipiente (c). De forma análoga, a massa do sistema após adicionar o objeto submerso pode ser determinado por uma expressão que iguala a Força Normal (N’) à Força Peso (P’):

\[ N' = P' \]
\[ N' = m'_{r+l} \cdot g \]
\[\therefore m'_{r+l} = \frac{N'}{g} \]

Como também foi utilizado uma balança para essa etapa, também pode-se simplificar a obtenção do sistema após a adição do objeto pela simples leitura do valor registrado na balança.

\begin{figure}[H]
    \centering
    \includegraphics[scale=0.8]{images/Experimento2.1.png}
    \caption{Leitura da balança após incluir um sólido de volume desconhecido no recipiente com água.}
\end{figure}

Através do diagrama de forças do recipiente, com o líquido na situação em que o corpo está submerso, obtém-se as seguintes expressões:

\[ E = N' - m_{r+l} \cdot g \]
\[ \rho_1 \cdot V_s \cdot g = (m'_{r+l} - m_{r+l}) \cdot g \]
\[\therefore V_s = \frac{(m'_{r+l} - m_{r+l})}{\rho_1} \]

Como o líquido em questão no qual o sólido foi submerso é a água, cuja densidade é igual a $\rho_{agua}$ = 1,0 g/c$m^3$, a expressão acima pode ser sintetizada pela simples diferença entre as duas leituras da balança, ou seja, o Volume do sólido submerso ($V_s$) é igual a:

\[ V_s = m'_{r+l} - m_{r+l} \]

Depois de calculado o volume, será determinado a densidade desse sólido submerso ($\rho_{solido}$). Para isso, será determinado a massa do sólido diretamente pela leitura na balança. 

\begin{figure}[H]
    \centering
    \includegraphics[scale=0.8]{images/Experimento2.3.png}
    \caption{ Leitura da balança após incluir o sólido.}
\end{figure}

Com esse valor e com o volume do sólido ($V_s$) calculado na etapa anterior, será determinado a Densidade do sólido pela seguinte expressão:

\[\rho_{solido} = \frac{m_{solido}}{V_s} \]

Após o cálculo, será feito a comparação da densidade obtida com o valor tabelado para, então, determinar de que material é feito o sólido.



% =============== EXPERIMENTO 3 ===================== %

\subsection{Determinação do volume e da densidade de um sólido
utilizando um Areômetro de Nicholson}

Judá


% =============== EXPERIMENTO 4 ===================== %

\subsection{Determinação da densidade de um líquido utilizando o
Areômetro de Nicholson}

Para os nossos cálculos, utilizaremos os seguintes dados que foram disponibilizados no vídeo da aula, em conjunto com um valor da densidade da água disponibilizado pela apostila:

\begin{table}[H]
    \centering
    \begin{tabular}{ |M{5cm}||M{2cm}||M{2cm}||M{2cm}|  }
        \hline
        \textbf{O que foi medido} & \textbf{Valor} & \textbf{Incerteza} & \textbf{Unidade}\\
        \hline
        Massa Aerômetro + Pesos ($m_{ar} + m_t$)         & 135,96    & $\pm$ 0,01 & g\\
        Massa dos Pesos na água ($m_t$)                     & 34,30     & $\pm$ 0,01 & g\\
        Massa dos Pesos no líquido desconhecido ($m_t'$)    & 52,21     & $\pm$ 0,01 & g\\
        \hline
        Densidade da água a 25°C ($\rho _{agua}$)           & 1,000 & - & g/$cm^3$\\
        \hline
    \end{tabular}
    \caption{Dados coletados para o experimento de aferição da densidade}
\end{table}

Assim sendo, podemos começar os cálculos. Primeiro vamos determinar o volume do aerômetro:

\[ V_{ar} = \frac{m_{ar} + m_t}{\rho _{agua}} \]
\[ V_{ar} = \frac{135,96}{1,000} = 135,96 (cm^3) \]

\[ \delta V_{ar} = \frac{\delta m_{ar} + \delta m_t}{\rho _{agua}} \]
\[ \delta V_{ar} = \frac{0,01 + 0,01}{1} = 0,02 (cm^3) \]

Com esse dado em mãos, conseguiremos assim determinar a densidade do fluido desconhecido:

\[ \rho _x = \rho _{agua} + \frac{m_t - m_t'}{V_{ar}} \]
\[ \rho _x = 1,000 + \frac{34,30 - 52,21}{135,96} = 0,868270078 (g/cm^3) \]

\[ \delta \rho _x = \rho _{agua} \cdot \frac{(\delta m_t + \delta m_t') \cdot V_{ar} + \delta V_{ar} \cdot (m_t - m_t')}{V_{ar}^2} \]
\[ \delta \rho _x = 1,000 \cdot \frac{(0,01 + 0,01) \cdot 135,96 + 0,02 \cdot (34,30 - 52,21)}{135,96^2} \]
\[ \delta \rho _x = 0,00012772434 (g/cm^3) \]

Dessa forma, ajustando os algarismos significativos teremos que a densidade do fluido desconhecido é:

\[ \mathbf{\therefore \rho _x = 0,8683 \pm 0,0001 (g/cm^3)} \]

Comparando esse resultado com uma tabela online, podemos dizer que o líquido transparente com a densidade mais próxima do nosso resultado seria o \textit{benzeno} ($\rho _{benzeno} = 0.876 g/cm^3$).\\

Entretanto, como o vídeo deixou uma dica sobre o que estava dentro da proveta, uma garrafa de álcool de limpeza. Ou seja, era para termos encontrado um valor próximo da densidade específica do etanol ($\rho _{etanol} = 0.789 g/cm^3$), portanto o nosso valor encontrado \textbf{não foi compatível com o esperado}.\\

Provavelmente ocorreram alguns problemas com o experimento. Talvez o álcool tenha evaporado consideravelmente, aumentando a concentração de água nele, levando a um aumento da densidade. Ou ainda, seja uma característica do álcool 70, que possui 30\% de água, e essa quantidade leva naturalmente a densidade ser diferente da do álcool puro ($\rho _{etanol}$).

