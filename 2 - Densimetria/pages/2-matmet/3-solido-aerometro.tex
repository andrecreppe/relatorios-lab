\subsection{Determinação do volume e da densidade de um sólido
utilizando um Areômetro de Nicholson}

Para essa prática vamos utilizar um Areômetro de Nicholson, que é formado por um cilindro oco de metal, ao qual estão presos dois pratos: um na parte de cima do cilindro, e outro na parte de baixo, como mostrado na figura a seguir.\\

\begin{figure}[H]
    \centering
    \includegraphics[scale=1.0]{images/Areômetro 1.png}
    \caption{Representação da estrutura de um Areômetro de Nicholson}
\end{figure}

O volume de areômetro será denominado $V_{ar}$ e seu peso total $P_{ar}$.\\

Uma haste o cilindro ao prato superior conta com uma marcação que referencia a medida, e é chamada de "traço de afloramento". O "afloramento" é obtido quando o aparato se encontra submerso em um líquido, em equilíbrio hidrostático, com o "traço de afloramento" coincidindo com o nível do fluido.\\

\begin{figure}[H]
    \centering
    \includegraphics[scale=1.0]{images/Areômetro 2.png}
    \caption{Representação do experimento para determinar o volume de um sólido utilizando um areômetro}
\end{figure}

Após termos entendido do que é constituído o Areômetro de Nicholson, vamos utilizá-lo para calcular o volume - e consequentemente - a densidade de um sólido desconhecido.

Para o cálculo do volume do sólido, primeiramente precisamos determinar a massa dele $m_s$, com a ajuda de uma balança digital. Colocamos então, o sólido no prato superior do areômetro e vamos adicionando uma massa $m_a$ até que o "afloramento" seja atingido - por conta do equilíbrio hidrostático em água. Mediremos a massa adicionada $m_a$, e anotaremos o valor. 
Dessa forma, a equação do equilíbrio hidrostático é:

\[ (m_s + m_a)g + P_{ar} = \rho _{agua} g V_{ar} \]

Então retiramos o sólido do prato superior, e o colocamos no prato inferior (que ficará submerso). Agora, adicionaremos uma massa $m_a'$ no prato superior até que ocorra o "afloramento". O equilíbrio hidrostático nos dá uma nova equação:

\[ (m_s + m_a')g + P_{ar} = \rho _{agua} g (V_{ar} + V_s) \]

Vamos combinar as duas equações acima, para obter o volume do sólido e sua densidade:

\[ V_s = \frac{(m_a' - m_a)}{\rho _{agua}} \]

e

\[ \rho _s = \frac{m_s}{V_s} = \frac{m_s}{(m_a' - m_a)} \rho _{agua} \]\\

Podemos dizer que a força de empuxo sobre o sólido é basicamente a diferença entre os pesos necessários para que ocorra o "afloramento" com o sólido no prato de cima ou no prato de baixo, pois quando ele está dentro do líquido, o empuxo sobre ele demanda uma massa $m_a'$ maior para compensá-lo. \\


 

