\newpage
\section{Conclusões}

% =============== EXPERIMENTO 1 ===================== %

\subsection{Princípio de Arquimedes}

Maia

% =============== EXPERIMENTO 2 ===================== %

\subsection{Determinação do volume e da densidade de um sólido com
uma balança}

Maia

% =============== EXPERIMENTO 3 ===================== %

\subsection{Determinação do volume e da densidade de um sólido
utilizando um Areômetro de Nicholson}

Após realizarmos todos as etapas das aferições utilizando o Areômetro de Nicholson, pudemos encontrar uma densidade do sólido $\rho _s = 8,61 \pm 0,07 g/cm^3$ . Ao procurarmos em uma tabela de densidades, o valor que melhor representou a densidade do nosso sólido desconhecido, foi o material latão, que tem uma densidade tabelada de $\rho_l = 8,56 g/cm^3$ . \\

Comparamos os valores e confirmamos que os valores dessas duas densidades são \textbf{compatíveis}, checando que a densidade do latão está dentro do intervalo de incertezas da densidade do sólido - densidade experimental - como expresso no gráfico da figura 10.\\

Desse modo, pudemos comprovar a eficácia do nosso experimento, mostrando que o Areômetro de Nicholson é um método seguro de calcular volumes - e, consequentemente, densidades - de sólidos desconhecidos, porém, pode ser usado para o cálculo da densidade de líquidos desconhecidos (como feito por nós no experimento 4).\\

% =============== EXPERIMENTO 4 ===================== %

\subsection{Determinação da densidade de um líquido utilizando o
Areômetro de Nicholson}

Creppe

