\newpage
\section{Conclusões}

% =============== EXPERIMENTO 2 ===================== %

\subsection{Determinação do volume e da densidade de um sólido com
uma balança}

Após determinarmos a densidade do sólido utilizando o princípio de Arquimedes na balança, chegamos a resultados satisfatórios. Nesse experimento, por meio das massas iniciais e finais do cilindro imerso, respectivamente $m_{r+l} = (83,70 \pm 0,01) g$ e $m'_{r+l} = (86,23 \pm 0,01) g$, realizadas na balança, calculamos seu volume, $V_s = 2,53 cm^3$,e logo em seguida sua densidade $\rho_s = (8,50 \pm 0,07) g/cm^3$.

A partir desse valor calculado para a densidade do sólido imerso, realizamos uma comparação com valores tabelados para determinar o material que constituía o cilindro em questão. Dessa forma, sabendo que a densidade do latão é $\rho_l = 8,56 g/cm^3$, analisamos se a densidade do latão estava dentro do intervalo de incertezas do nosso sólido desconhecido. Concluímos, portanto, que ambas as densidades são \textbf{compatíveis} e que o sólido imerso é constituído de latão.

% =============== EXPERIMENTO 3 ===================== %

\subsection{Determinação do volume e da densidade de um sólido
utilizando um Areômetro de Nicholson}

Após realizarmos todos as etapas das aferições utilizando o Areômetro de Nicholson, pudemos encontrar uma densidade do sólido $\rho _s = 8,61 \pm 0,07 g/cm^3$ . Ao procurarmos em uma tabela de densidades, o valor que melhor representou a densidade do nosso sólido desconhecido, foi o material latão, que tem uma densidade tabelada de $\rho_l = 8,56 g/cm^3$ . \\

Comparamos os valores e confirmamos que os valores dessas duas densidades são \textbf{compatíveis}, checando que a densidade do latão está dentro do intervalo de incertezas da densidade do sólido - densidade experimental - como expresso no gráfico da figura 11.\\

Desse modo, pudemos comprovar a eficácia do nosso experimento, mostrando que o Areômetro de Nicholson é um método seguro de calcular volumes - e, consequentemente, densidades - de sólidos desconhecidos, porém, pode ser usado para o cálculo da densidade de líquidos desconhecidos (como feito por nós no experimento 4).\\

% =============== EXPERIMENTO 4 ===================== %

\subsection{Determinação da densidade de um líquido utilizando o
Areômetro de Nicholson}

No final do experimento descobrimos que o valor da densidade desconhecida ($\rho _x$) foi muito próximo do valor da água salgada encontrada no \textbf{mar morto}. Se o líquido que foi pesado realmente simulasse a densidade do mar morto, podemos dizer que o nosso resultado \textbf{não foi compatível com o esperado}.\\

Isso pode ter sido fruto principalmente do fato de que o Aerômetro não é o equipamento recomendado para medir densidade de líquidos, uma vez que ele apresenta uma precisão menor do que para medir sólidos.

% No final do experimento descobrimos que o valor da densidade desconhecida ($\rho _x$) não foi compatível experimentalmente com o valor esperado, que era a do álcool 70\%. Chegamos a conclusão que isso pode ter sido fruto de diversos fatores, como a evaporação do álcool que testamos e o fato dele conter 30\% de água em sua composição, levando a um aumento na densidade geral.\\

% Mas o que acreditamos que foi o principal fator é o fato de que adaptamos um instrumento com outra finalidade pra medir essa grandeza. O Aerômetro não foi pensado para medir densidade de líquidos, e por causa disso a sua precisão diminui, o que pode ter levado ao nosso resultado desviar do padrão.
