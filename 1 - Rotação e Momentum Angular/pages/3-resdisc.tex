\newpage
\section{Resultados e discussão}

% =============== EXPERIMENTO 1 ===================== %

\subsection{Determinação do Momento de Inércia de um disco}

Baseado no vídeo disponibilizado, logo abaixo temos uma tabela que reuni os valores medidos experimentalmente e que utilizaremos nos nossos cálculos para esse experimento:

\begin{table}[H]
    \centering
    \begin{tabular}{ |p{5cm}||p{2cm}||p{2cm}||p{2cm}|  }
        \hline
        \textbf{O que foi medido} & \textbf{Valor} & \textbf{Incerteza} & \textbf{Unidade}\\
        \hline
        Massa do anel (m\textsubscript{a}) & 0.9238 & \SI{\pm 0.0001} & kg\\
        Massa do disco (m\textsubscript{d}) & 0.4707 & \SI{\pm 0.0001} & kg\\
        Massa do eixo (m\textsubscript{e}) & 0.12125 & \SI{\pm 0.00001} & kg\\
        Comprimento do eixo (R) & 0.0060 & \SI{\pm 0.0001} & m\\
        Raio menor do disco (R\textsubscript{d1}) & 0.0060 & \SI{\pm 0.0001} & m\\
        Raio maior do disco (R\textsubscript{d}) & 0.0625 & \SI{\pm 0.0001} & m\\
        Raio menor do anel (R\textsubscript{a1}) & 0.0625 & \SI{\pm 0.0001} & m\\
        Raio maior do anel (R\textsubscript{a}) & 0.0660 & \SI{\pm 0.0001} & m\\
        \hline
    \end{tabular}
    \caption{Dimensões do Disco de Maxwell}
\end{table}

\begin{table}[H]
    \centering
    \begin{tabular}{ |p{5cm}||p{2cm}||p{2cm}||p{2cm}|  }
        \hline
        \textbf{O que foi medido} & \textbf{Valor} & \textbf{Incerteza} & \textbf{Unidade}\\
        \hline
        Altura (h) & 0.467 & \SI{\pm 0.001} & m\\
        Tempo de queda 1 (t\textsubscript{b1}) & 0.312 & \SI{\pm 0.001} & s\\
        Tempo de queda 2 (t\textsubscript{b2}) & 0.311 & \SI{\pm 0.001} & s\\
        Tempo de queda 3 (t\textsubscript{b3}) & 0.312 & \SI{\pm 0.001} & s\\
        \hline
    \end{tabular}
    \caption{Dados experimentais do Disco de Maxwell}
\end{table}

- momento de inércia das partes + incerteza\\
- momento para cada lanceamento (3x) + incerteza => determinar a media?\\
- compare os resultados

% =============== EXPERIMENTO 2 ===================== %

\subsection{Choques Rotacionais}

aaaaaaaaa

% =============== EXPERIMENTO 3 ===================== %

\subsection{Conservação do Momento Angular}

aaaaaaaaaa

% =============== EXPERIMENTO 4 ===================== %

\subsection{Precessão do Giroscópio}

Baseado no video disponibilizado para esse experimento os dados que temos disponíveis são os seguintes:

\begin{table}[H]
    \centering
    \begin{tabular}{ |p{5cm}||p{2cm}||p{2cm}||p{2cm}|  }
        \hline
        \textbf{O que foi medido} & \textbf{Valor} & \textbf{Incerteza} & \textbf{Unidade}\\
        \hline
        Meio eixo (D) & 0.0565 & \SI{\pm 0.0001} & m\\
        Diâmetro do eixo (d\textsubscript{eixo}) & 0.0070 & \SI{\pm 0.0001} & m\\
        Raio externo (R\textsubscript{1}) & 0.0702 & \SI{\pm 0.0001} & m\\
        Raio interno (R\textsubscript{2}) & 0.0625 & \SI{\pm 0.0001} & m\\
        Largura externa (l\textsubscript{1}) & 0.0203 & \SI{\pm 0.0001} & m\\
        Largura interna (l\textsubscript{2}) & 0.0043 & \SI{\pm 0.0001} & m\\
        Massa do conjunto (M) & 1.116 & \SI{\pm 0.001} & kg\\
        Densidade do material ($\rho _{aco}$) & 8000.0 & - & $\frac{kg}{m^3}$\\
        \hline
    \end{tabular}
    \caption{Dados físicos do Giroscópio}
\end{table}

\begin{table}[H]
    \centering
    \begin{tabular}{ |p{5cm}||p{2cm}||p{2cm}||p{2cm}|  }
        \hline
        \textbf{O que foi medido} & \textbf{Valor} & \textbf{Incerteza} & \textbf{Unidade}\\
        \hline
        Tempo 1-1 (t\textsubscript{11}) & 0.072 & \SI{\pm 0.001} & s\\
        Tempo 1-2 (t\textsubscript{12}) & 0.135 & \SI{\pm 0.001} & s\\
        Tempo 1-3 (t\textsubscript{13}) & 0.194 & \SI{\pm 0.001} & s\\
        Velocidade 1 ($\omega _1$) & 271.2 & \SI{\pm 0.1} & $\frac{rad}{s}$\\
        \hline
        Tempo 2-1 (t\textsubscript{21}) & 0.073 & \SI{\pm 0.001} & s\\
        Tempo 2-2 (t\textsubscript{22}) & 0.138 & \SI{\pm 0.001} & s\\
        Tempo 2-3 (t\textsubscript{23}) & 0.198 & \SI{\pm 0.001} & s\\
        Velocidade 2 ($\omega _2$) & 255.6 & \SI{\pm 0.1} & $\frac{rad}{s}$\\
        \hline
        Tempo 3-1 (t\textsubscript{31}) & 0.076 & \SI{\pm 0.001} & s\\
        Tempo 3-2 (t\textsubscript{32}) & 0.144 & \SI{\pm 0.001} & s\\
        Tempo 3-3 (t\textsubscript{33}) & 0.206 & \SI{\pm 0.001} & s\\
        Velocidade 3 ($\omega _3$) & 308.2 & \SI{\pm 0.1} & $\frac{rad}{s}$\\
        \hline
        Tempo 4-1 (t\textsubscript{41}) & 0.074 & \SI{\pm 0.001} & s\\
        Tempo 4-2 (t\textsubscript{42}) & 0.142 & \SI{\pm 0.001} & s\\
        Tempo 4-3 (t\textsubscript{43}) & 0.205 & \SI{\pm 0.001} & s\\
        Velocidade 4 ($\omega _4$) & 256.4 & \SI{\pm 0.1} & $\frac{rad}{s}$\\
        \hline
    \end{tabular}
    \caption{Dados experimentais do Giroscópio}
\end{table}

- Estimar a massa do giroscópio\\
- Calcular o momento de inercia dele\\
- Frequência estimada (4x) \\
\\
- Frequência real (4x)\\
- Determinar diretamente (pelo tempo) a frequência de precessão\\
- Comparar as frequências
