\newpage
\section{Resultados e discussão}

Baseado nos vídeos, logo abaixo temos uma tabela que reuni os valores medidos experimentalmente e que utilizaremos nos nossos cálculos:

\begin{table}[H]
    \centering
    \begin{tabular}{ |p{5cm}||p{2cm}||p{2cm}||p{2cm}|  }
        \hline
        \textbf{O que foi medido} & \textbf{Valor} & \textbf{Incerteza} & \textbf{Unidade}\\
        \hline
        Massa do anel (m\textsubscript{a}) & 0.9238 & \SI{\pm 0.0001} & kg\\
        Massa do disco (m\textsubscript{d}) & 0.4707 & \SI{\pm 0.0001} & kg\\
        Massa do eixo (m\textsubscript{e}) & 0.12125 & \SI{\pm 0.00001} & kg\\
        Comprimento do eixo (R) & 0.0060 & \SI{\pm 0.0001} & m\\
        Raio menor do disco (R\textsubscript{d1}) & 0.0060 & \SI{\pm 0.0001} & m\\
        Raio maior do disco (R\textsubscript{d}) & 0.0625 & \SI{\pm 0.0001} & m\\
        Raio menor do anel (R\textsubscript{a1}) & 0.0625 & \SI{\pm 0.0001} & m\\
        Raio maior do anel (R\textsubscript{a}) & 0.0660 & \SI{\pm 0.0001} & m\\
        \hline
    \end{tabular}
    \caption{Dimensões do Disco de Maxwell}
\end{table}

\begin{table}[H]
    \centering
    \begin{tabular}{ |p{5cm}||p{2cm}||p{2cm}||p{2cm}|  }
        \hline
        \textbf{O que foi medido} & \textbf{Valor} & \textbf{Incerteza} & \textbf{Unidade}\\
        \hline
        Altura (h) & 0.467 & \SI{\pm 0.001} & m\\
        Tempo de queda 1 (t\textsubscript{b1}) & 0.312 & \SI{\pm 0.001} & s\\
        Tempo de queda 2 (t\textsubscript{b2}) & 0.311 & \SI{\pm 0.001} & s\\
        Tempo de queda 3 (t\textsubscript{b3}) & 0.312 & \SI{\pm 0.001} & s\\
        \hline
    \end{tabular}
    \caption{Dados experimentais do Disco de Maxwell}
\end{table}

% =============== EXPERIMENTO 1 ===================== %

\subsection{Determinação do Momento de Inércia de um disco}
aaaaaaaaa

% =============== EXPERIMENTO 2 ===================== %

\subsection{Choques Rotacionais}
aaaaaaaaa

% =============== EXPERIMENTO 3 ===================== %

\subsection{Conservação do Momento Angular}
aaaaaaaaaa

% =============== EXPERIMENTO 4 ===================== %

\subsection{Precessão do Giroscópio}
aaaaaaaaa
