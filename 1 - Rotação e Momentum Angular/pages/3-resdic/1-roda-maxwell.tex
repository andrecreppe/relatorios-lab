\subsection{Determinação do Momento de Inércia de um disco}

Baseado no vídeo disponibilizado, logo abaixo temos uma tabela que reuni os valores medidos experimentalmente e que utilizaremos nos nossos cálculos para determinar I fisicamente:

\begin{table}[H]
    \centering
    \begin{tabular}{ |M{5cm}||M{2cm}||M{2cm}||M{2cm}|  }
        \hline
        \textbf{O que foi medido} & \textbf{Valor} & \textbf{Incerteza} & \textbf{Unidade}\\
        \hline
        Massa do anel (m\textsubscript{a})          & 0,9238    & $\pm$ 0,0001  & kg\\
        Massa do disco (m\textsubscript{d})         & 0,4707    & $\pm$ 0,0001  & kg\\
        Massa do eixo (m\textsubscript{e})          & 0,12125   & $\pm$ 0,00001 & kg\\
        Raio do eixo (R\textsubscript{e})           & 0,0060    & $\pm$ 0,0001  & m\\
        Raio menor do disco (R\textsubscript{d1})   & 0,0060    & $\pm$ 0,0001  & m\\
        Raio maior do disco (R\textsubscript{d})    & 0,0625    & $\pm$ 0,0001  & m\\
        Raio menor do anel (R\textsubscript{a1})    & 0,0625    & $\pm$ 0,0001  & m\\
        Raio maior do anel (R\textsubscript{a})     & 0,0760    & $\pm$ 0,0001  & m\\
        \hline
    \end{tabular}
    \caption{Dimensões e propriedades físicas do Disco de Maxwell}
\end{table}

Utilizado esses valores, podemos calcular os momentos de inércia:

\[I_E = \frac{1}{2} \cdot m_e \cdot R_e^2\]
\[I_E = \frac{1}{2} \cdot 0,12125 \cdot (0,0060)^2 = 0,0000021825 (kg \cdot m^2)\]

\[I_D = \frac{1}{2} \cdot m_d \cdot (R_{d1}^2 + R_d^2)\]
\[I_D = \frac{1}{2} \cdot 0,4707 \cdot [(0,0060)^2 + (0,0625)^2] = 0,0009278085 (kg \cdot m^2) \]

\[I_A = \frac{1}{2} \cdot m_a \cdot (R_{a1}^2 + R_a^2)\]
\[I_A = \frac{1}{2} \cdot 0,9238 \cdot [(0,0625)^2 + (0,0760)^2] = 0,0044722313 (kg \cdot m^2) \]

E consequentemente as incertezas desses valores serão:

\[\Delta I_E = \frac{1}{2} \left( \Delta m_e \cdot R_e^2 + 2 \cdot R_e \cdot \Delta R_e \cdot m_e \right)\]
\[\Delta I_E = \frac{1}{2} \left( 0,00001 \cdot (0,0060)^2 + 2 \cdot 0,0060 \cdot 0,0001 \cdot 0,0001 \right) = 0,000000000078 (kg \cdot m^2)\]

\[\Delta I_D = \frac{1}{2} 
    \left[ 
        \Delta m_d \cdot (R_{d1}^2 + R_d^2) + 
        \left( 2 \cdot R_d \cdot \Delta R_d + 2 \cdot R_{d1} \cdot \Delta R_{d1} \right) \cdot m_d
    \right]
\]
\[\Delta I_D = \frac{1}{2} 
    \left[ 
        0,0001 \cdot ((0,0060)^2 + (0,0625)^2) + 
        \left( 2 \cdot 0,0625 \cdot 0,0001 + 2 \cdot 0,0060 \cdot 0,0001 \right) \cdot 0,4707
    \right]
\]
\[\Delta I_D = 0,0000034214 (kg \cdot m^2)\]

\[\Delta I_A = \frac{1}{2} 
    \left[ 
        \Delta m_a \cdot (R_{a1}^2 + R_a^2) + 
        \left( 2 \cdot R_a \cdot \Delta R_a + 2 \cdot R_{a1} \cdot \Delta R_{a1} \right) \cdot m_a
    \right]
\]
\[\Delta I_A = \frac{1}{2} 
    \left[ 
        0,0001 \cdot ((0,0625)^2 + (0,0760)^2) + 
        \left( 2 \cdot 0,0760 \cdot 0,0001 + 2 \cdot 0,0625 \cdot 0,0001 \right) \cdot 0,9238
    \right]
\]
\[\Delta I_A = 0,0000132787 (kg \cdot m^2)\]

Com os momentos de cada peça, podemos agora somá-los:

\[I_F = I_E + I_D + I_A\]
\[I_F = 0,0000021825 + 0,0009278085 + 0,0044722313 = 0,0054022223 (kg \cdot m^2)\]

\[\Delta I_F = \Delta I_E + \Delta I_A + \Delta I_D\]
\[I_F = 0,000000000078 + 0,0000034214 + 0,0000132787 = 0,000016700178 (kg \cdot m^2)\]

Assim, ajustando os algarismos temos que o momento de inércia da Roda de Maxwell do laboratório calculado fisicamente vale:

\[\therefore \mathbf{I_F = 0,00540 \pm 0,00002 (kg \cdot m^2)}\]

Uma parte do experimento foi concluída. Agora, vamos determinar indiretamente o valor da incerteza utilizando o método da queda. Os dados obtidos estão na tabela abaixo (os intervalos foram calculados subtraindo o tempo final do inicial).

\begin{table}[H]
    \centering
    \begin{tabular}{ |M{6cm}||M{2cm}||M{2cm}||M{2cm}|  }
        \hline
        \textbf{O que foi medido} & \textbf{Valor} & \textbf{Incerteza} & \textbf{Unidade}\\
        \hline
        Altura (h)                                      & 0,467 & $\pm$ 0,001 & m\\
        Tempo de queda inicial 1 (t\textsubscript{1i})  & 0,219 & $\pm$ 0,001 & s\\
        Tempo de queda final 1 (t\textsubscript{1f})    & 0,312 & $\pm$ 0,001 & s\\
        Intervalo 1 (t\textsubscript{1})                & 0,093 & $\pm$ 0,002 & s\\
        Tempo de queda inicial 2 (t\textsubscript{2i})  & 0,219 & $\pm$ 0,001 & s\\
        Tempo de queda final 2 (t\textsubscript{2f})    & 0,311 & $\pm$ 0,001 & s\\
        Intervalo 2 (t\textsubscript{2})                & 0,092 & $\pm$ 0,002 & s\\
        Tempo de queda inicial 3 (t\textsubscript{3i})  & 0,222 & $\pm$ 0,001 & s\\
        Tempo de queda final 3 (t\textsubscript{3f})    & 0,312 & $\pm$ 0,001 & s\\
        Intervalo 3 (t\textsubscript{3})                & 0,090 & $\pm$ 0,002 & s\\
        \hline
    \end{tabular}
    \caption{Dados experimentais da queda do Disco de Maxwell}
\end{table}

Primeiramente vamos determinar o tempo médio de queda utilizando a média aritmética entre os intervalos calculados:

\[\overline{t}_b = \frac{t_{b1} + t_{b2} + t_{b3}}{3} = \frac{0,093 + 0,092 + 0,900}{3} = 0,0916666667 (s)\]

\[\Delta \overline{t}_b = \sqrt{\frac{\sum_{i=1}^{3} (t_i - \overline{t})^2}{3}}\]
\[\Delta \overline{t}_b = \sqrt{\frac{(0,093-0,916666667)^2 + (0,092-0,916666667)^2 + (0,900-0,916666667)^2}{3}}\]
\[\Delta \overline{t}_b = 0,001247219 (s)\]

Ajustando as incertezas do tempo teremos:

\[\therefore \overline{t}_b = 0,092 \pm 0,001 (s)\]

Com o tempo médio, podemos aplicar todos os valores na nossa fórmula do momento para queda do Disco de Maxwell:

\[m = m_a + m_d + m_e = 0,9238 + 0,4707 + 0,12125 = 1,51575 (kg)\]

\[I_E = \left( \frac{g \overline{t}_b^2}{2h} - 1 \right) m r_e^2\]
\[I_E = \left( \frac{9,80665 \cdot (0,092)^2}{2 \cdot 0,467} - 1 \right) \cdot 1,51575 \cdot (0,0060)^2\]
\[I_E = \left(0,91113117 \right) \cdot 0,008754972 = 0,0049718 (kg \cdot m^2)\]

E as incertezas para esses valores calculados acima se dão por:

\[\Delta m = \Delta m_a + \Delta m_d + \Delta m_e = 0,0001 + 0,0001 + 0,00001 = 0,00021 (kg)\]

\[
    \Delta P = \frac{g}{2} 
    \left( 
        \frac
        {\overline{t}_b^2 \cdot \Delta h + 2 \cdot \overline{t}_b \cdot \Delta \overline{t}_b \cdot h}
        {h^2} 
    \right)
\]
\[
    \Delta P = \frac{9,80665}{2} \left(\frac{(0,092)^2 \cdot 0,001 + 2 \cdot 0,092 \cdot 0,001 \cdot 0,467}{(0,467)^2} \right) = 0,002122228
\]
\[\Delta Q = 2mr \Delta r + \Delta m r^2\]
\[\Delta Q = 2 \cdot 1,51575 \cdot 0,0001 + 0,00021 \cdot (0,0060)^2 = 0,00003032256\]

\[\Delta I_E = \Delta P \cdot Q + \Delta Q \cdot P\]
\[\Delta I_E = 0,002122228 \cdot 0,008754972 + 0,00003032256 \cdot 0,91113117 = 0,0000462079 (kg \cdot m^2)\]

Finalmente, ajustando os algarismos significativos teremos então:

\[\mathbf{I_E = 0,00497 \pm 0,00005 (kg \cdot m^2)}\]

Com os dois resultados em mãos, podemos comparar os resultados. Poderemos dizer que ambos os valores calculados são \textit{equivalentes} experimentalmente quando a desigualdade abaixo for satisfeita:

\[| I_F - I_E | < 2 (\Delta I_F + \Delta I_E)\]

Dessa forma:

\[| 0,00540 - 0,00497 | < 2 \cdot (0,00002 + 0,00005)\]
\[\mathbf{0,00043 < 0,00014 \Rightarrow FALSO}\]

Como podemos observar, a desigualdade não é satisfeita, por isso podemos afirmar que os valores de momento de inercia encontrados \textbf{não foram compatíveis experimentalmente}. \\

Isso quer dizer que houve algum parâmetro nas medições que fez com que os resultados se divergissem entre si. Alguns fatores que podem ter contribuído para essa diferença podem ter sido:

\begin{enumerate}
    \item \textbf{Erros na medição do tempo} $\xrightarrow{}$ Como nós que determinamos o tempo baseado na nossa capacidade de pausar o vídeo no momento certo, isso pode ter causado uma variação nos resultados que, caso tivéssemos utilizado um cronômetro automático, isso não teria acontecido;
    \item \textbf{Atrito} $\xrightarrow{}$ Nessa situação estamos considerando nosso sistema como se ele fosse ideal, mas não é. Por mais que seja pequeno, existe o atrito entre o barbante e o eixo enquanto ele cai, em conjunto com o arraste, além da dissipação de energia pela elasticidade da corda que está amarrada na roda.
\end{enumerate}
