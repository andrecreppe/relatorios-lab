\subsection{Precessão do Giroscópio}

Baseado no video disponibilizado para esse experimento os dados que temos disponíveis são os seguintes:

\begin{table}[H]
    \centering
    \begin{tabular}{ |M{5cm}||M{2cm}||M{2cm}||M{2cm}|  }
        \hline
        \textbf{O que foi medido} & \textbf{Valor} & \textbf{Incerteza} & \textbf{Unidade}\\
        \hline
        Meio eixo (D)                               & 0,0565    & $\pm$ 0,0001  & m\\
        Diâmetro do eixo (d\textsubscript{eixo})    & 0,0070    & $\pm$ 0,0001  & m\\
        Raio externo (R\textsubscript{1})           & 0,0702    & $\pm$ 0,0001  & m\\
        Raio interno (R\textsubscript{2})           & 0,0625    & $\pm$ 0,0001  & m\\
        Largura externa (l\textsubscript{1})        & 0,0203    & $\pm$ 0,0001  & m\\
        Largura interna (l\textsubscript{2})        & 0,0043    & $\pm$ 0,0001  & m\\
        Massa do conjunto (M)                       & 1,116     & $\pm$ 0,001   & kg\\
        Densidade do material ($\rho _{aco}$)       & 8000,0    & -             & kg/$m^3$\\
        \hline
    \end{tabular}
    \caption{Dados físicos do Giroscópio}
\end{table}

Assim sendo, podemos começar a calcular as massas de cada peça do giroscópio:

\[ r_{eixo} = \frac{d_{eixo}}{2} \Rightarrow m_e = (2 D \pi r_{eixo}^2) \cdot \rho_{aco} \]
\[ r_{eixo} = \frac{0,0070}{2} \Rightarrow m_e = (2 \cdot 0,0565 \cdot \pi \cdot (0,0035)^2) \cdot 8000,0 = 0,34789997 (kg)\]

\[ m_d = (l_2 \pi R_2^2) \cdot \rho_{aco} \]
\[ m_d = (0,0043 \pi (0,0625)^2) \cdot 8000,0 = 0,4221515 (kg) \]

\[ m_a = (l_1 \pi R_1^2) \cdot \rho_{aco} \]
\[ m_a = (0,0203 \pi (0,0702 - 0,0625)^2) \cdot 8000,0 = 0,302494 (kg) \]

Observe que se somarmos os valores acima obteremos 1,0725459 kg, que é bem próximo da massa do conjunto M, indicando que nossos cálculos estão no caminho certo (o resto da massa está no pedaço de alumínio que foi pedido para desconsiderarmos). Propagando as incertezas agora teremos:

\[\Delta r_{eixo} = \frac{\Delta d_{eixo}}{2} = \frac{0,0001}{2} = 0,00005 (m)\]

\[ \Delta m_e = \rho_{aco} \cdot 2 \pi \left( \Delta D \cdot r_{eixo}^2 + D \cdot 2 \cdot r_{eixo} \cdot \Delta r_{eixo} \right)\]
\[ \Delta m_e = 8000,0 \cdot 2 \pi \left(0,0001 \cdot (0,0035)^2 + 0,0565 \cdot 2 \cdot 0,0035 \cdot 0,00005 \right) = 0,0010555751 (kg)\]

\[ \Delta m_d = \rho_{aco} \cdot \pi \left( \Delta l_2 \cdot R_2^2 + l_2 \cdot 2 \cdot R_2 \cdot \Delta R_2 \right) \]
\[ \Delta m_d = 8000,0 \cdot \pi \left( 0,0001 \cdot (0,0625)^2 + 0,0043 \cdot 2 \cdot 0,0625 \cdot 0,0001 \right) = 0,011168362 (kg) \]

\[ \Delta m_a = \rho_{aco} \cdot \pi \left( \Delta l_1 \cdot R_1^2 + l_1 \cdot 2 \cdot R_1 \cdot \Delta R_1 \right) \]
\[ \Delta m_a = 8000,0 \cdot \pi \left( 0,0001 \cdot (0,0702)^2 + 0,0203 \cdot 2 \cdot 0,0702 \cdot 0,0001 \right) = 0,019548648 (kg)\]

Prosseguindo, podemos determinar os momentos de inércia de cada componente e consequentemente do giroscópio todo:

\[I_E = \frac{1}{2} \cdot m_e \cdot R_e^2\]
\[I_E = \frac{1}{2} \cdot 0,34789997 \cdot (0,0035)^2 = 0,0000021309 (kg \cdot m^2)\]

\[I_D = \frac{1}{2} \cdot m_d \cdot (R_2^2 + R_e^2)\]
\[I_D = \frac{1}{2} \cdot 0,4221515 \cdot [(0,0625)^2 + (0,0035)^2] = 0,0008271003 (kg \cdot m^2) \]

\[I_A = \frac{1}{2} \cdot m_a \cdot (R_1^2 + R_2^2)\]
\[I_A = \frac{1}{2} \cdot 0,302494 \cdot [(0,0702)^2 + (0,0625)^2] = 0,0013361599 (kg \cdot m^2) \]\\

Propagando as incertezas:

\[\Delta I_E = \frac{1}{2} \left( \Delta m_e \cdot R_e^2 + 2 \cdot R_e \cdot \Delta R_e \cdot m_e \right)\]
\[\Delta I_E = \frac{1}{2} \left( 0,0010555751 \cdot (0,0035)^2 + 2 \cdot 0,0035 \cdot 0,00005 \cdot 0,34789997 \right)\]
\[\Delta I_E = 0,0000000369 (kg \cdot m^2)\]

\[\Delta I_D = \frac{1}{2} 
    \left[ 
        \Delta m_d \cdot (R_2^2 + R_e^2) + 
        \left( 2 \cdot R_e \cdot \Delta R_e + 2 \cdot R_2 \cdot \Delta R_2 \right) \cdot m_d
    \right]
\]
\[\Delta I_D = \frac{1}{2} 
    [ 
        0,011168362 \cdot ((0,0625)^2 + (0,0035)^2) + ( 2 \cdot 0,0035 \cdot 0,00005
\]\[
         + 2 \cdot 0,0625 \cdot 0,0001) \cdot 0,4221515
    ]
\]
\[\Delta I_D = 0,00002459394 (kg \cdot m^2)\]

\[\Delta I_A = \frac{1}{2} 
    \left[ 
        \Delta m_a \cdot (R_1^2 + R_2^2) + 
        \left( 2 \cdot R_1 \cdot \Delta R_1 + 2 \cdot R_2 \cdot \Delta R_2 \right) \cdot m_a
    \right]
\]
\[\Delta I_A = \frac{1}{2} 
    [ 
        0,019548648 \cdot ((0,0702)^2 + (0,0625)^2) + 
        (2 \cdot 0,0702 \cdot 0,0001 
\]\[
        + 2 \cdot 0,0625 \cdot 0,0001) \cdot 0,302494
    ]
\]
\[\Delta I_A = 0,0000903633 (kg \cdot m^2)\]\\

Assim sendo, para \textit{I} teremos:

\[I_T = I_E + I_D + I_A\]
\[I_T = 0,0010555751 + 0,011168362 + 0,019548648 = 0,00216539107 (kg \cdot m^2)\]

\[\Delta I_T = \Delta I_E + \Delta I_A + \Delta I_D\]
\[\Delta I_T = 0,0000000369 + 0,00002459394 + 0,0000903633 = 0,00011499414 (kg \cdot m^2)\]

Com o momento de inércia em mãos, podemos prosseguir para a próxima parte. Para isso, precisaremos dos outros dados do experimento, disponíveis na tabela abaixo (os valores para as velocidades angulares $\omega _i$ )

\begin{table}[H]
    \centering
    \begin{tabular}{ |M{5cm}||M{2cm}||M{2cm}||M{2cm}|  }
        \hline
        \textbf{O que foi medido} & \textbf{Valor} & \textbf{Incerteza} & \textbf{Unidade}\\
        \hline
        Tempo 1 (t\textsubscript{1})        & 19,4      & $\pm$ 0,1     & s\\
        Velocidade 1 ($\omega _1$)          & 271,2     & $\pm$ 0,1     & rad/s\\
        \hline
        Tempo 2 (t\textsubscript{2})        & 19,8      & $\pm$ 0,1     & s\\
        Velocidade 2 ($\omega _2$)          & 255,6     & $\pm$ 0,1     & rad/s\\
        \hline
        Tempo 3 (t\textsubscript{3})        & 20,6      & $\pm$ 0,1     & s\\
        Velocidade 3 ($\omega _3$)          & 308,2     & $\pm$ 0,1     & rad/s\\
        \hline
        Tempo 4 (t\textsubscript{4})        & 20,5      & $\pm$ 0,1     & s\\
        Velocidade 4 ($\omega _4$)          & 256,4     & $\pm$ 0,1     & rad/s\\
        \hline
    \end{tabular}
    \caption{Dados experimentais do Giroscópio}
\end{table}

Finalmente podemos calcular o valor de $\Omega _e$, para isso vamos repetir a formula 4 vezes, uma para cada velocidade angular medida com o tacômetro:

\[ \Omega _{E1} = \frac{MgD}{I \omega _1} \]
\[ \Omega _{E1} = \frac{1,116 \cdot 9,80665 \cdot 0,0565}{0,00216539107 \cdot 271,2} = \frac{0,618348509}{0,857254058} = 1,05294889 (rad/s)\]

\[ \Omega _{E2} = \frac{MgD}{I \omega _2} \]
\[ \Omega _{E2} = \frac{1,116 \cdot 9,80665 \cdot 0,0565}{0,00216539107 \cdot 255,6} = \frac{0,618348509}{0,553473958} = 1,11721377 (rad/s)\]

\[ \Omega _{E3} = \frac{MgD}{I \omega _3} \]
\[ \Omega _{E3} = \frac{1,116 \cdot 9,80665 \cdot 0,0565}{0,00216539107 \cdot 308,2} = \frac{0,618348509}{0,667373528} = 0,926654036 (rad/s)\]

\[ \Omega _{E4} = \frac{MgD}{I \omega _4} \]
\[ \Omega _{E4} = \frac{1,116 \cdot 9,80665 \cdot 0,0565}{0,00216539107 \cdot 256,4} = \frac{0,618348509}{0,555206270} = 1,11372753 (rad/s)\]

Então, a média dos valores de $\Omega _E$ obtidos experimentalmente é:

\[ \overline{\Omega _E} = \frac{\Omega _{E1} + \Omega _{E2} + \Omega _{E3} + \Omega _{E4}}{4} \]
\[ \overline{\Omega _E} = \frac{1,05294889 + 1,11721377 + 0,926654036 + 1,11372753}{4} = 1,052636055 (rad/s) \]

Dessa forma, podemos propagar as incertezas pelo método do desvio padrão:

\[ \Delta \overline{\Omega _E} = \sqrt{\frac{\sum_{i=1}^{4} (\Omega _{Ei} - \overline{\Omega _E})^2}{4}}\]
\[ \Delta \overline{\Omega _E} = \sqrt{\frac{0.251651305}{4}} = 0.250824294 (rad/s)\]

Ajustando os valores, teremos então que:

\[ \therefore \mathbf{\Omega _E = 1,1 \pm 0,3 (rad/s)}\]

Agora na segunda parte desse experimento, vamos utilizar os valores de tempo $t_i$ para determinar o $\Omega _d$:

\[ \Omega _{D1} = \frac{6 \pi}{t_1} = \frac{6 \pi}{19,4} = 0,9716265939 (rad/s)\]

\[ \Omega _{D2} = \frac{6 \pi}{t_2} = \frac{6 \pi}{19,8} = 0,9519977738 (rad/s)\]

\[ \Omega _{D3} = \frac{6 \pi}{t_3} = \frac{6 \pi}{20,6} = 0,9150269865 (rad/s)\]

\[ \Omega _{D4} = \frac{6 \pi}{t_4} = \frac{6 \pi}{20,5} = 0,9194905328 (rad/s)\]

Fazendo a média desses valores:

\[ \overline{\Omega _D} = \frac{\Omega _{D1} + \Omega _{D2} + \Omega _{D3} + \Omega _{D4}}{4} \]
\[ \overline{\Omega _D} = \frac{0,9716265939 + 0,9519977738 + 0,9150269865 + 0,9194905328}{4} = 0,93953547175 (rad/s) \]

E de forma análoga ao anterior, propagaremos as incertezas pelo utilizando o desvio padrão:

\[ \Delta \overline{\Omega _D} = \sqrt{\frac{\sum_{i=1}^{4} (\Omega _{Di} - \overline{\Omega _D})^2}{4}}\]
\[ \Delta \overline{\Omega _D} = \sqrt{\frac{0.058045566}{4}} = 0.120463238 (rad/s)\]

Ajustando teremos:

\[ \therefore \mathbf{\Omega _D = 0,9 \pm 0,1 (rad/s)}\]

Para analisar esses valores, vamos utilizar a mesma desigualdade emprega no experimento 1, ou seja, o absoluto da diferença entre $\Omega _E$ e $\Omega _D$ deve ser menor que o dobro da soma das suas respectivas incertezas:

\[ | 1,1 - 0,9 | < 2 \cdot (0,3 + 0,1) \]
\[ \mathbf{0,2 < 0,8 \Rightarrow VERDADEIRO} \]

Ou seja, os valores encontrados para a frequência de precessão do giroscópio foram \textbf{compatíveis experimentalmente}! Isso quer dizer que o nosso modelo matemático foi observado com uma certa exatidão na prática.

Alguns fatores como o atrito do eixo do giroscópio com a base de rotação, o arraste do ar fazendo a roda diminuir sua velocidade e mais importante, a dificuldade de pesar com exatidão cada componente para determinar o momento de inércia são alguns fatores que diminuíram a nossa precisão no resultado, mas mesmo assim não atrapalharam muito, indicando que fomos bem criteriosos na montagem e realização do experimento.

% E as incertezas para cada item (como o numerador N não varia, só precisamos calculá-lo uma vez):

% \[ \Delta N = g \cdot (M \cdot \Delta D + \Delta M \cdot D) \]
% \[ \Delta N = 9,80665 \cdot (1,116 \cdot 0,0001 + 0,001 \cdot 0,0565) = 0,0016484978\]

% \[ \Delta D_1 = \Delta I \cdot \omega _1 + I \cdot \Delta \omega _1\]
% \[ \Delta D_1 = (0,00011499414 \cdot 271,2) + (0,00216539107 \cdot 0,1) = 0,0314029499 \]

% \[ \Delta \Omega _{E1} = g \cdot \frac{(\Delta N \cdot D_1) + (N \cdot \Delta D_1)}{D_1^2} \]
% \[ \Delta \Omega _{E1} = 9,80665 \cdot \frac{(0,0016484978 \cdot 0,857254058) + (0,618348509 \cdot 0,0314029499)}{0,857254058^2} \]
% \[ \Delta \Omega _{E1} = 0,277798080 (rad/s) \]

% \[ \Delta D_2 = \Delta I \cdot \omega _2 + I \cdot \Delta \omega _2\]
% \[ \Delta D_2 = (0,00011499414 \cdot 255,6) + (0,00216539107 \cdot 0,1) = 0,029609041 \]

% \[ \Delta \Omega _{E2} = g \cdot \frac{(\Delta N \cdot D_2) + (N \cdot \Delta D_2)}{D_2^2} \]
% \[ \Delta \Omega _{E2} = 9,80665 \cdot \frac{(0,0016484978 \cdot 0,553473958) + (0,618348509 \cdot 0,029609041)}{0,553473958^2} \]
% \[ \Delta \Omega _{E2} = 0,615391973 (rad/s) \]

% \[ \Delta D_3 = \Delta I \cdot \omega _3 + I \cdot \Delta \omega _3\]
% \[ \Delta D_3 = (0,00011499414 \cdot 308,2) + (0,00216539107 \cdot 0,1) = 0,035657733 \]

% \[ \Delta \Omega _{E3} = g \cdot \frac{(\Delta N \cdot D_3) + (N \cdot \Delta D_3)}{D_3^2} \]
% \[ \Delta \Omega _{E3} = 9,80665 \cdot \frac{(0,0016484978 \cdot 0,667373528) + (0,618348509 \cdot 0,035657733)}{0,667373528^2} \]
% \[ \Delta \Omega _{E3} = 0,5937280747 (rad/s) \]

% \[ \Delta D_4 = \Delta I \cdot \omega _4 + I \cdot \Delta \omega _4\]
% \[ \Delta D_4 = (0,00011499414 \cdot 256,4) + (0,00216539107 \cdot 0,1) = 0,029701037 \]

% \[ \Delta \Omega _{E4} = g \cdot \frac{(\Delta N \cdot D_4) + (N \cdot \Delta D_4)}{D_4^2} \]
% \[ \Delta \Omega _{E4} = 9,80665 \cdot \frac{(0,0016484978 \cdot 0,555206270) + (0,618348509 \cdot 0,029701037)}{0,555206270^2} \]
% \[ \Delta \Omega _{E4} = 0,613391965 (rad/s) \]
