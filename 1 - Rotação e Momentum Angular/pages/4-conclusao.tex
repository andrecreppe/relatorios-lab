\newpage
\section{Conclusões}

% =============== EXPERIMENTO 1 ===================== %

\subsection{Determinação do Momento de Inércia de um disco}

Após realizarmos o experimento, calculando todos os valores necessários para a obtenção do momentos de inércias e suas incertezas - encontrados fisicamente e experimentalmente - comparamos os valores para testar sua \textit{equivalência}, pela fórmula:

\[| I_F - I_E | < 2 (\Delta I_F + \Delta I_E)\]

Como a desigualdade para nossos valores é \textit{falsa}, concluímos que os momentos de inércia encontrados por nós - fisicamente ($I_F$) e experimentalmente ($I_E$) - \textbf{não são compatíveis}.\\

Dessa forma, imaginamos que algum fator na medições influenciou para que os resultados fossem incompatíveis. Alguns parâmetros que podem ter causado essa diferença nos resultados, podendo citar:

\begin{enumerate}
    \item \textbf{Erros na medição do tempo} $\xrightarrow{}$ A determinação do tempo foi feita por nós pausando o vídeo no momento desejado, uma maneira pouco precisa de estimar tempos muito pequenos. Assim, pode ter ocorrido uma diferença em nossos valores, causada pela imprecisão desse método, algo que talvez não acontecesse se tivéssemos usado um método mais preciso para medir os tempos - como cronômetros digitais;  
    
    \item \textbf{Atrito} $\xrightarrow{}$ Em nosso experimento, tomamos o sistema como ideal - não levando em conta as dissipações de energia -, quando na verdade, existem pequenas dissipações de diversas naturezas. Entre os fatores que causaram variações nos valores, podemos destacar o atrito entre o fio e o eixo enquanto ele cai, o arraste do ar em contato com a peça, além da dissipação na forma de energia elástica - pela elasticidade da corda que está amarrada à roda. 
\end{enumerate}

% =============== EXPERIMENTO 2 ===================== %

\subsection{Choques Rotacionais}

Após a análise do comportamento de um sistema formado por duas peças em processo de choque rotacional, chegamos a resultados satisfatórios. Por meio das dimensões geométricas e da massa da peça, calculamos o Momento de Inércia da Peça 2 ($I_2$) e chegamos ao seguinte resultado: $I_2 = (0,00522 \pm 0,00002) kg.m^2$.

A partir da análise dos dados obtidos no experimento anterior (Roda de Maxwell) e da coleta das velocidades angulares no vídeo, construímos um gráfico da Energia de Rotação em função do Tempo, que apresentou um comportamento linear. Esse comportamento linear foi suficiente para determinarmos aproximadamente a variação de energia durante o movimento de rotação da peça no eixo. Essa perda de energia foi justificada pela dissipação da energia devido ao atrito do eixo com a peça.

Após isso, conhecendo apenas as velocidades angulares iniciais e finais para três repetições, e o Momento de Inércia da Peça 2, determinamos indiretamente o Momento de Inércia da Peça 1 ($I_1$): $I_1 = (0,0061 \pm 0,0005) kg.m^2$.

Com os valores calculados para o Momento de Inércia da Peça 1 e da Peça 2, utilizamos as velocidades angulares obtidas no vídeo para determinar as Energias Cinéticas de Rotação Inicial ($E_{ci}$) e Final ($E_{cf}$) com suas respectivas incertezas: $E_{ci} = 1,3 \pm 0,6 J$ e $E_{cf}0,7 \pm 0,4 J$. 

Através do cálculo da Variação Relativa, determinamos em porcentagem a quantidade aproximada de energia dissipada durante o choque rotacional: $\Delta E_c (\%) = 46 (\%)$. Isso significa que durante o experimento não, houve conservação de energia já que ao comparar a energia cinética rotacional inicial e final houve uma diferença considerável. Esse resultado já era esperado, sendo justificado pelo gráfico desenhado em etapas anteriores, que mostram a perda de energia durante a rotação da peça em torno do eixo, e pelo fato do experimento representar um choque perfeitamente inelástico que por característica não envolve conservação de energia.

Por fim, comparamos os resultados obtidos para o Momento de Inércia da Peça 1 dos experimentos da Roda de Maxwell e do Choque Rotacional. Concluímos que houve equivalência entre os valores obtidos de $I_1$ quando comparado o experimento do Choque Rotacional com a obtenção direta pelas características geométricas da peça 1. Isso confirma a confiabilidade dos dois métodos, sendo que cada um possui suas vantagens e desvantagens próprias. Entretanto, quando comparado o método do Choque Rotacional com o método a partir do lançamento da peça de uma altura H (Roda de Maxwell), não chegamos a um resultado suficiente para afirmarmos se houve ou não equivalência dos valores.

% =============== EXPERIMENTO 3 ===================== %

\subsection{Conservação do Momento Angular}

Com o estudo da conservação do momento angular, vimos casos de conservação total em sistemas - nos quais não existe torques externos-, e casos de conservação em apenas um eixo - como no caso do banco giratório com roda de bicicleta-, podendo entender melhor como se dão os saltos, ou as variações de velocidade de rotação, e por que isso ocorre. 

Na ausência de torques externos, $L_{sis}$ é constante:

\[T_{ext sist} = \frac{\partial L_{sis}}{\partial t} = 0 \Rightarrow L_{sis} = cte\]

Compreendemos também, como variações nos raios de rotação, alteram a distribuição de massa dos corpos, interferindo diretamente em seu  momento de inércia inicial $I_1$, fazendo com que ele mude para um momento de inércia final $I_2$, mudando diretamente sua velocidade angular de $\omega_1$ para $\omega_2$.

Na conservação:
\[I_1 \omega_1 = I_2 \omega_2\]

Após os três experimentos dessa seção, podemos destacar a importância da conservação do momento angular para inúmeras situações do nosso cotidiano, exaltando a importância de entender a parte física dos movimentos para compreender melhor o funcionamento de coisas simples, como giros e saltos. 

% =============== EXPERIMENTO 4 ===================== %

\subsection{Precessão do Giroscópio}

Fechando com chave de ouro nossa prática, o experimento em que determinamos a frequência de precessão do giroscópio, calculando todos os valores necessários para a obtenção dessa grandeza e suas incertezas - encontradas de forma direta e indireta - comparamos os valores para testar sua \textit{equivalência}, pela fórmula de comparação:

\[| \Omega _E - \Omega _D | < 2 (\Delta \Omega _E + \Delta \Omega _D)\]

Como a desigualdade para nossos valores é \textit{verdadeira}, concluímos que as frequências de precessão encontradas por nós - direta ($\Omega _D$) e indireta ($\Omega _E$) - \textbf{são compatíveis}.\\

Dessa forma, mesmo com fatores que influenciariam de forma negativa no experimento, obtivemos uma boa precisão nos resultados, e o fator de vivermos num mundo não idealizado, com várias forças dissipativas não foi páreo para um bom cuidado e boas habilidades matemáticas.
