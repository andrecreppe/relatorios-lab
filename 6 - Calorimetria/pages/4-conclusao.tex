\newpage
\section{Conclusões}

% =============== EXPERIMENTO 1 ===================== %

\subsection{Determinação da capacidade térmica do calorímetro}

Desse modo, após o experimento do cálculo empírico da capacidade térmica de um calorímetro, podemos concluir que o $C$ do calorímetro utilizado por nós é:

\[ \mathbf{ C = 18 \pm 3 \ \  cal/^\circ C}\]

Mostramos que a capacidade térmica do calorímetro será diferente da capacidade térmica apenas do copo de alumínio, pois o calorímetro é formado majoritariamente por isopor - que tem um calor específico maior que o do alumínio -, e também, por plástico - que tem valores que diferem daquele do alumínio, dependendo do tipo de polímero utilizado.\\

Assim, conseguimos perceber que a capacidade térmica do calorímetro deve ser maior que o do copo de alumínio, e isso de fato ocorre:

\[\mathbf{ C_{copo} = 11,297 \pm 0,002 \ \ cal/^\circ C \ \ < \ \ C = 18 \pm 3 \ \ cal/^\circ C}\]

Logo, conseguimos encontrar o valor da capacidade térmica - que seria utilizado em nossos outros experimentos -, valor esse, que condiz com nossa hipótese de que deveria ser maior que o $C_{copo}$. Como encontramos um valor plausível para $C$, conseguimos confirmar que a equação que deduzimos para o cálculo da capacidade térmica faz sentido e está correta.

% =============== EXPERIMENTO 2 ===================== %

\subsection{Determinação do calor específico de um metal}

Nessa segunda prática pudemos testar uma técnica muito prática para se descobrir o calor específico de materiais, especialmente metais, utilizando apenas água, fogo, termômetros e um calorímetro.

Com ela, conseguimos pegar um pedaço de metal desconhecido e obter obter que o seu calor específico foi de:

\[ \mathbf{c_m = 0,089 \pm 0,004 } \]

E comparando esse valor com uma tabela de referencia fornecida pela apostila, em conjunto com a informação visual da cor desse objeto obtida pelo vídeo, pudemos concluir com boa precisão que ele era de fato um pedaço de \textbf{Cobre}, pois o nosso resultado foi \textbf{compatível experimentalmente} com tal valor tabelado ($c_{cu} = 0,093$).

Também provamos que, mesmo desconsiderando outras formas de troca de calor com o ambiente que podem ter acontecido entre os passos do experimento, nossos cálculos e a habilidade dos profissionais que executaram foram boas o suficiente para fazer com que isso não fosse um grande problema e atrapalhasse no resultado.

% =============== EXPERIMENTO 3 ===================== %

\subsection{Determinação do calor latente de condensação da água}

No terceiro experimento, nosso objetivo foi determinar o calor latente de condensação da água, $L_c$. Para isso, inicialmente registramos os valores das massas e temperaturas coletadas que seriam utilizados na equação de $L_c$ descrita na metodologia. Com base na massa total do sistema ($m_f$) e na massa inicial de água fria ($m_1$), calculamos a massa de água condensada: $m_2 = 19,54 \pm 0,02$.

Após isso, aplicamos os valores na equação e chegamos ao seguinte valor para o calor latente de condensação da água:

\[ \mathbf{L_C = (-570 \pm 10) \ \  cal/g}\]

Então, sabendo que o valor de referência para o calor latente de condensação da água é $L_{c-ref}= -539 cal/g$, podemos concluir que chegamos a um resultado satisfatório, porém não equivalente dado que o valor obtido experimentalmente com sua respectiva incerteza não abrange o intervalo do valor de referência. 

Por fim, realizamos uma análise acerca do sinal obtido para o calor latente de condensação da água. Sabendo que  condensação é uma transformação física provocada pela perda de calor ($Q < 0$), na fórmula de calorimetria $Q = mL$, a variável L (coeficiente de calor latente) será negativa. Como o valor obtido tem sinal negativo, confirmamos tal levantamento teórico.
