\newpage
\section{Conclusões}

% =============== EXPERIMENTO 1 ===================== %

\subsection{Determinação da capacidade térmica do calorímetro}

Desse modo, após o experimento do cálculo empírico da capacidade térmica de um calorímetro, podemos concluir que o $C$ do calorímetro utilizado por nós é:
\[ \mathbf{C = 18,25 \ \  cal/^\circ C}\]

Mostramos que a capacidade térmica do calorímetro será diferente da capacidade térmica apenas do copo de alumínio, pois o calorímetro é formado majoritariamente por isopor - que tem um calor específico maior que o do alumínio -, e também, por plástico - que tem valores que diferem daquele do alumínio, dependendo do tipo de polímero utilizado.\\

Assim, conseguimos perceber que a capacidade térmica do calorímetro deve ser maior que o do copo de alumínio, e isso de fato ocorre:
\[\mathbf{ C_{copo} = 11,30 \ \ cal/^\circ C \ \ < \ \ C = 18,25 \ \ cal/^\circ C}\]

Logo, conseguimos encontrar o valor da capacidade térmica - que seria utilizado em nossos outros experimentos -, valor esse, que condiz com nossa hipótese de que deveria ser maior que o $C_{copo}$. Como encontramos um valor plausível para $C$, conseguimos confirmar que a equação que deduzimos para o cálculo da capacidade térmica faz sentido e está correta.

% =============== EXPERIMENTO 2 ===================== %

\subsection{Determinação do calor específico de um metal}

Creppe

% =============== EXPERIMENTO 3 ===================== %

\subsection{Determinação do calor latente de condensação da água}

No terceiro experimento, nosso objetivo foi determinar o calor latente de condensação da água, $L_c$. Para isso, inicialmente registramos os valores das massas e temperaturas coletadas que seriam utilizados na equação de $L_c$ descrita na metodologia. Com base na massa total do sistema ($m_f$) e na massa inicial de água fria ($m_1$), calculamos a massa de água condensada: $m_2 = 19,54 \pm 0,02$.

Após isso, aplicamos os valores na equação e chegamos ao seguinte valor para o calor latente de condensação da água:

\[ \mathbf{L_C = -572,47190 \ \  cal/g}\]

Então, sabendo que o valor de referência para o calor latente de condensação da água é $L_{c-ref}= -539 cal/g$, podemos concluir que chegamos a um resultado satisfatório dado que o valor obtido experimentalmente é equivalente ao valor de referência. Portanto, concluímos que o experimento apresentou resultados satisfatórios.

Por fim, realizamos uma análise acerca do sinal obtido para o calor latente de condensação da água. Sabendo que  condensação é uma transformação física provocada pela perda de calor ($Q < 0$), na fórmula de calorimetria $Q = mL$, a variável L (coeficiente de calor latente) será negativa. Como o valor obtido tem sinal negativo, confirmamos tal levantamento teórico.
