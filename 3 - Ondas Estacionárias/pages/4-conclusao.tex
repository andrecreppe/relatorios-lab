\newpage
\section{Conclusões}

% =============== EXPERIMENTO 1 ===================== %
\subsection{Ondas estacionárias na corda}

Como vimos no experimento, provamos que determinar a densidade linear de uma corda utilizando o método envolvendo o MHS, mais especificamente as ondas estacionárias, é equivalente a usar uma balança e régua. Tudo isso pois a desigualdade abaixo foi satisfeita, indicando que os valores encontrados são compatíveis experimentalmente:

\[ |\mu_e - \mu_d| < 2 \cdot (\delta \mu_e + \delta \mu_d) \]
\[ \mathbf{0,000006 < 0,000011 \xrightarrow{} VERDADEIRO} \]

Mais uma vez, comprovamos a relação entre as propriedades físicas e confirmamos a teoria, mostrando que ela realmente determina propriedades físicas de corpos reais.

% =============== EXPERIMENTO 2 ===================== %
\subsection{Ondas estacionárias de som: geração de harmônicos
em função da frequência f}

aaaaaaaaaaaaaaa

% =============== EXPERIMENTO 3 ===================== %
\subsection{Ondas estacionárias de som: geração de harmônicos em
função do comprimento L}

aaaaaaaaaaaaaaa

% =============== EXPERIMENTO 4 ===================== %
\subsection{Ondas estacionárias de som em um gás nobre
desconhecido}

aaaaaaaaaaaaaaa

% =============== EXPERIMENTO 5 ===================== %
\subsection{Ondas estacionárias de som no tubo de Rubens}

aaaaaaaaaaaaaaa
