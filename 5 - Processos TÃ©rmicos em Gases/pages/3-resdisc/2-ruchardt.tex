\subsection{Fator $\gamma$ do ar: Método de Rüchardt}

Primeiramente, o experimento foi montado pelos responsáveis do laboratório e disponibilizados por meio das gravações de vídeo. Assim sendo, nos pudemos extrair os seguintes dados e convertendo-os para o SI temos:

\begin{table}[H]
    \centering
    \begin{tabular}{ |c||c||c||c|  }
        \hline
        \textbf{O que foi medido} & \textbf{Valor} & \textbf{Incerteza} & \textbf{Unidade}\\
        \hline 
         Massa da bolinha (m)       & 0,01672   & 0,00001   & kg\\
         Diâmetro do tubo (d)       & 0,0160    & 0,0001    & m\\
         Volume do recipiente (V)   & 0,0104    & 0,0001    & $m^3$\\
         \hline
         Período de 4 oscilações ($T_4$) & 4,82 & 0,01 & s\\
         \hline
         Pressão atmosférica (P) & 92125,8 & 0,1 & Pa\\
        \hline
    \end{tabular}
    \caption{Dados físicos e dimensões do experimento de Rüchardt} 
\end{table}

Com os dados em mãos podemos calcular tudo que precisamos. Antes de tudo, entretanto, é necessário obter o período de apenas uma oscilação (T), o que é simples, pois basta dividir $T_4$ por 4:

\[ T = \frac{T_4}{4} = \frac{4,82}{4} = 1,205 s \]
\[ \delta T = \frac{\delta T_4}{4} = \frac{0.01}{4} = 0,0025 s \]

Além disso, precisamos da área do tubo onde a bolinha estava suspensa. Mais uma vez, esse é um resultado bem fácil de ser obtido, já que temos o diâmetro do tubo e a seção do tubo é um círculo:

\[ r = \frac{d}{2} = \frac{0,0160}{2} = 0,008 m \]
\[ A = \pi r^2 = 3,14159 \cdot (0,008)^2 = 0.0002010619 m^2 \]

\[ \delta r = \frac{\delta d}{2} = \frac{0,0001}{2} = 0,00005 m \]
\[ \delta A = \pi (\delta r)^2 = 3,14159 \cdot (0,00005)^2 = 0.00000000785 m^2 \]

Agora sim podemos calcular o nosso fator $\gamma$ do ar atmosférico, que é uma tarefa bem direta, pois basta inserirmos os dados experimentais na equação geral:

\[ N = m \cdot V \]
\[ N = 0,01672 \cdot 0,0104 = 0,000173888 \]

\[ AT = A^2 \cdot T^2 \]
\[ AT =(0,0002010619)^2 \cdot (1,205)^2 = 0,0000000586994 \]

\[ D = P \cdot AT \]
\[ D = 92125,8 \cdot 0,0000000586994 = 0,0054077307 \]

\[ \gamma = 4\pi^2 \cdot \frac{N}{D} \]
\[ \gamma = 4\pi^2 \cdot \frac{0,000173888}{0,0054077307} = 1,26944617 \]

Agora, propagando as incertezas para essas medidas temos:

\[ \delta N = \delta m \cdot V + m \cdot \delta V \]
\[ \delta N = 0,00001 \cdot 0,0104 + 0,01672 \cdot 0,0001 = 0,000001776 \]

\[ \delta a = 2 \cdot A \cdot \delta A \]
\[ \delta a = 2 \cdot 0.0002010619 \cdot 0.00000000785 = 0,00000000000315667 \]

\[ \delta t = 2 \cdot T \cdot \delta T \]
\[ \delta t = 2 \cdot 1,205 \cdot 0,0025 = 0,006025 \]

\[ \delta AT = A^2 \cdot \delta t + \delta a \cdot T^2 \]
\[ \delta AT = (0,0002010619)^2 \cdot 0,006025 + 0,00000000000315667 \cdot (1,205)^2 \]
\[ \delta AT = 0,0000000002481496 \]

\[ \delta D = P \cdot \delta AT + \delta P \cdot AT \]
\[ \delta D = 92125,8 \cdot 0,0000000002481496 + 0,1 \cdot 0,0000000586994 \]
\[ \delta D = 0,00002286684 \]

\[ \delta \gamma = 4\pi^2 \cdot \frac{N \cdot \delta D + \delta N \cdot D}{D^2} \]
\[ \delta \gamma = 4\pi^2 \cdot \frac{0,000173888 \cdot 0,00002286684 + 0,000001776 \cdot 0,0054077307}{0,0054077307^2} \]
\[ \delta \gamma = 0,005835689998 \]

Assim sendo, ajustando os algarismos significativos teremos que o fator $\gamma$ do ar atmosférico equivale a:

\[ \therefore \mathbf{\gamma = (1,269 \pm 0,006)} \]

Podemos notar que esse valor se aproxima muito do esperado para um \textbf{gás poliatômico} ($\gamma_p \approx$ 1,30). Entretanto, nossa atmosfera é 99\% composta de gases diatômicos (78\% $N_2$, 21\% $O_2$) e o fator $\gamma$ para esses gases é maior ($\gamma_d \approx$ 1,40).\\

Isso quer dizer que o nosso experimento \textbf{não foi compatível experimentalmente}, uma vez que a distribuição do valor era para englobar $\gamma_d$ mas não o fez. Por causa disso, devemos tentar entender quais foram os pontos que podem ter diminuído a precisão da nossa prática: 

\begin{enumerate}
    \item \textbf{Processo termodinâmico real}: Nós vivemos em um mundo onde os elementos que compõem o aparato experimental não são ideais. Ou seja, o frasco retira calor do ar dentro dele a fim de atingir um equilíbrio térmico durante o processo termodinâmico. Mesmo que o vidro seja um bom isolante, ele ainda sim realiza essa troca, e essa energia retirada faz a diferença nos nossos cálculos.
    \item \textbf{Atrito}: A bolinha que oscila tem que estar em contato com a parede do tubo vertical para criar uma vedação. Isso quer dizer que ela esta em contato com outra superfície e portanto existe um atrito entre eles. Mesmo que ela seja polida ele existe, e pode ser um dos fatores que retiraram energia do sistema. Além disso, temos uma força de arrasto no momento da oscilação, que também ajuda a converter a energia cinética da bolinha em calor.
\end{enumerate}

Outro fator que poderia ter influenciado os resultados seria se o ar do laboratório com uma concentração maior de gases poliatômicos, sendo o mais provável deles o Dióxido de Carbono ($CO_2$). Entretanto, isso seria impossível, visto que o professor não conseguiria gravar os vídeos por não conseguir respirar.
