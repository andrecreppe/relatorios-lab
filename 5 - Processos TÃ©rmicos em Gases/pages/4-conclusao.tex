\newpage
\section{Conclusões}

% =============== EXPERIMENTO 1 ===================== %
\subsection{Fator $\gamma$ do ar: Método de Clément – Desormes}

Juda

% =============== EXPERIMENTO 2 ===================== %
\subsection{Fator $\gamma$ do ar: Método de Rüchardt}

Pudemos ver que esse método de Rüchardt para determinar o fator $\gamma$ do ar é muito inusitado, uma vez que um movimento harmônico não parece algo que pode estar relacionado com alguma característica de um processo termodinâmico. Mas de qualquer forma, provamos que é possível relacionar essas duas áreas da física e conseguimos obter um valor experimental para o $\gamma$ do ar do laboratório de física, sendo ele:

\[ \mathbf{\gamma = (1,269 \pm 0,006)} \]

Entretanto, também comprovamos que ele \textbf{não foi compatível experimentalmente}, uma vez que não foi equivalente ao esperado para gases diatômicos ($\gamma_d \approx$ 1,40), que seria o correto, pois nossa atmosfera é composta praticamente de gases desse tipo.\\

Tal diferença provavelmente vêm do fato de que vivemos num mundo real, onde existe atrito entre superfícies (bolinha e tudo vertical) e o recipiente troca calor com o gás que estamos realizando processo termodinâmico. Esses fatores fazem com que energia seja retirada do sistema, impossibilitando medir com muita precisão o valor de $\gamma$.

\textit{}{FAZER A COMPARAÇÃO ENTRE OS DOIS MÉTODOS DO FATOR DO AR???}

% =============== EXPERIMENTO 3 ===================== %
\subsection{Zero absoluto: Determinação do zero absoluto utilizando um termômetro a
gás}

No terceiro experimento, nosso objetivo foi determinar o coeficiente $\beta$ de dilatação do gás à pressão constante utilizando de conceitos físicos básicos relacionados a transformações isovolumétricas. Após a construção de um gráfico relacionando a pressão do gás (cmHg) em função da temperatura (ºC), determinamos pelo método dos mínimos quadrados os valores dos coeficientes lineares (b) e dos coeficientes angulares (a), sendo iguais a $b = 62,1$ e $a = 0,222901$.\\

Após isso, estabelecemos uma relação entre os valores de $\beta$ e $P_0$ com os coeficientes encontrados anteriormente, sendo que o valor da pressão $P_0$ corresponde ao coeficiente linear e o valor coeficiente $\beta$ de dilatação dos gases ideais a volume constante corresponde ao $P_0$ dividido pelo coeficiente angular. Dessa forma, chegamos aos seguintes valores: $P_0 = 62,1 cmHg$ e $\beta = 0,003589$.\\

Em seguida, determinamos a equação P(T) que define a variação da pressão do gás em função da variação da temperatura: $P(T) = 0,222901 \cdot T + 62,1$. Após traçar a respectiva reta da equação, extrapolamos o gráfico até atingir o eixo x que correspondia exatamente ao ponto de temperatura zero absoluto. Portanto, para determinar o valor do zero absoluto do gráfico, utilizamos um conceito que define conceitualmente que no ponto de zero absoluto, a pressão do gás é igual a zero. Ao substituir na equação, obtivemos o seguinte valor para a temperatura de zero absoluto: $T_{Zero Absoluto}= -278,60 ºC$.\\

Por fim, sabendo que o valor de referência para a temperatura do ponto de zero absoluto é igual a $T_{Zero Absoluto Ref}= -273 ºC$, podemos concluir que chegamos a um resultado satisfatório dado que o valor obtido experimentalmente é muito próximo da referência. Portanto, concluímos que o experimento apresentou resultados satisfatórios.
