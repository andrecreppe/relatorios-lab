\newpage
\section{Conclusões}

% =============== EXPERIMENTO 1 ===================== %
\subsection{Ondas estacionárias na corda}

Como vimos no experimento, provamos que determinar a densidade linear de uma corda utilizando o método envolvendo o MHS, mais especificamente as ondas estacionárias, é equivalente a usar uma balança e régua. Tudo isso pois a desigualdade abaixo foi satisfeita, indicando que os valores encontrados são compatíveis experimentalmente:

\[ |\mu_e - \mu_d| < 2 \cdot (\delta \mu_e + \delta \mu_d) \]
\[ 0,000006 < 0,000011 \xrightarrow{} VERDADEIRO \]

Mais uma vez, comprovamos a relação entre as propriedades físicas e confirmamos a teoria, mostrando que ela realmente determina propriedades físicas de corpos reais.

% =============== EXPERIMENTO 2 ===================== %
\subsection{Ondas estacionárias de som: geração de harmônicos
em função da frequência f}

Após a avaliação da geração de harmônicos em função da variação da frequência $f$, chegamos a resultados satisfatórios, mas com ressalvas. Nesse experimento, por meio do comprimento L fixado ($L = 0,110 \pm 0,001 m$), realizamos cinco (5) leituras no osciloscópio de frequências que correspondessem ao momento de maior pressão nas ondas. \\

Baseado em uma análise gráfica, aplicamos o método dos mínimos quadrados na reta $F_n$ versus $\frac{n}{2L}$ para chegarmos no valor do coeficiente angular ( a = $(0,327 \pm 0,001)\cdot 10^3$). Conforme discutido ao longo do relatório, pela equação de ondas estacionárias, pudemos atribuir à velocidade da onda o valor encontrado para coeficiente angular. Dessa forma, a velocidade do som no ar para o experimento em questão foi igual a $v = (327 \pm 1) m/s$.\\

A partir desse valor calculado para a velocidade do som no ar, realizamos uma comparação com valores de referência para determinar se a velocidade experimental era correspondente. Sabendo que a velocidade do som no ar para a temperatura local no laboratório (25ºC) é igual a $343 m/s$, concluímos que o valor experimental é \textbf{não compatível}. Entretanto, observamos que os valores são relativamente próximos, o que reafirma nossa tese de que possíveis erros sistemáticos ao longo do experimento, como a leitura das frequências, podem ter ocasionados dados dispersos, influenciando no resultado final. \\

Além disso, realizamos uma análise para determinar qual seria a frequência esperada para o harmônico fundamental (n=1) com a velocidade do som no ar de referência. Realizado os cálculos, chegamos ao valor de $f'_1 = (1,5591)kHz$, diferente em alguns algarismos do valor utilizado no experimento ($f_1 = 1,5634 kHz$). Dessa forma, podemos concluir que tal diferença pode ter influenciado também no resultado final encontrado para a velocidade experimental, já que as frequências para cada harmônico estão inclusos diretamente nos cálculos e na construção gráfica.\\

Por fim, também analisamos o valor encontrado para o coeficiente linear no método dos mínimos quadrados. Conforme a equação de onda estacionária ($f_n = v \cdot \frac{n}{2L}$), o coeficiente angular deveria assumir um valor nulo já que não há um termo independente na equação. Entretanto, chegamos a um valor diferente de zero para o coeficiente: $b = (0,08 \pm 0,01) \cdot 10^3$. Assim, podemos apontar essa informação como outro fator suficiente para interferir nos resultados experimentais encontrados.\\

Dessa forma, conclui-se que o experimento não ofereceu condições ou dados suficientes para a obtenção de um valor dentro do intervalo de referência.\\

% =============== EXPERIMENTO 3 ===================== %
\subsection{Ondas estacionárias de som: geração de harmônicos em
função do comprimento L}

No terceiro experimento, realizamos uma análise da geração de harmônicos em função da variação do comprimento L da coluna de ar do dispositivo e,após isso, chegamos a resultados satisfatórios, mas com ressalvas. Nesse experimento, por meio da frequência f fixado ($f = 2,0004 kHz$), realizamos cinco (5) leituras no osciloscópios de comprimentos L que correspondessem ao momento de maior pressão nas ondas. \\

Inicialmente, com base na equação $L_n = n \cdot \frac{\lambda_n}{2}$, concluímos matematicamente que a diferença entre os comprimentos L sucessivos ($L_{n+1} - L_n$) é igual à metade do comprimento de onda, que inicialmente é desconhecido. Para determinar o valor do comprimento de onda $\lambda$ e comprovar tal levantamento, calculamos o valor dos comprimentos de onda para cada harmônico correspondente. Após isso, realizamos uma média aritmética simples e determinamos o valor de $\lambda$ com sua incerteza: $\lambda = (0,167 \pm 0,003)m$.\\

Com base no valor encontrado para o comprimento de onda, utilizamos a equação fundamental da onda e determinamos o valor da velocidade experimental do som no ar, correspondendo ao seguinte valor: $v = (334 \pm 6) m/s $. Sabendo que o valor de referência é $v = 343 m/s$, concluímos que o valor experimental é \textbf{não compatível}.\\

Como forma de complementar as análises, realizamos uma análise gráfica e aplicamos o método dos mínimos quadrados na reta $L_n$ versus $\frac{n}{2f}$ para chegarmos ao valor do coeficiente angular  a = $(0,347 \pm 0,001)\cdot 10^3$. Pela equação de ondas estacionárias, pudemos atribuir à velocidade da onda o valor encontrado para coeficiente angular. Dessa forma, utilizando o método dos mínimos quadrados, a velocidade do som no ar para o experimento em questão foi igual a $v = (347 \pm 1) m/s$.\\

Dessa forma, utilizando o método dos mínimos quadrados e calculando diretamente a partir da equação fundamental das ondas, chegamos a valores experimentais diferentes para a velocidade. Além de serem distintos, ambos os valores são \textbf{não compatíveis} com a velocidade de referência do som no ar. Assim, podemos concluir que erros sistemáticos, como a leitura da régua do dispositivo, e aproximações de valores ao longo dos cálculos levaram a uma dispersão de dados, interferindo nos resultados finais, assim como no experimento anterior. Portanto, conclui-se que o experimento não ofereceu condições ou dados suficientes para a obtenção de um valor dentro do intervalo de referência.\\

% =============== EXPERIMENTO 4 ===================== %
\subsection{Ondas estacionárias de som em um gás nobre
desconhecido}

Depois de calcular a velocidade experimental $v_{exp}$ no gás desconhecido, a comparamos com uma tabela de velocidades do som. Como o prof. Eduardo havia dito que se tratava de um gás nobre, nos ativemos mais a esses gases. \\

Comparamos a $v_{exp}$ com todos os gases nobres, e o Hélio foi aquele que teve uma velocidade mais próxima - $v_{exp} \approx$ 91,76\% de $v_{helio}$.

\[v_{exp} = 924 \pm 4 m/s \ e \ v_{helio} = 1007 m/s \  a \ 20^\circ C \]

Essa diferença nos valores da velocidade experimental $v_{exp}$ com o valor tabelado $v_{helio}$ pode ter ocorrido por alguns fatores, como aproximações em cálculos ou em valores de medições, aliados ao fato de que não temos um valor tabelado para a velocidade do som no hélio à exata temperatura da sala, e que tivemos que utilizar o valor mais próximo conhecido, que foi a velocidade à temperatura de 20ºC.\\  

Outro fator que nos levou a desconfiar do gás Hélio, foi sua facilidade de obtenção quando comparado aos demais gases nobres como Xenônio ou Kriptônio, sendo mais facil adquiri-lo para usar no experimento.\\

Por fim, concluímos que de fato o gás desconhecido nesse experimento era o gás Hélio.

% =============== EXPERIMENTO 5 ===================== %
\subsection{Ondas estacionárias de som no tubo de Rubens}

Após nossos cálculos, montamos tabelas que nos auxiliaram a enxergar melhor a situação. Fizemos um gráfico de $f_n$ versus $n/2L$ e, pudemos calcular a velocidade do som no GLP quente como sendo o coeficiente angular da melhor reta ajustada entre os pontos desse gráfico - utilizando o método dos mínimos quadrados.\\

O resultado obtido para a velocidade foi:
\[ \therefore v_{GLP} = 276 \pm 4 \ m/s \]

O gás GLP é majoritariamente formado por uma mistura dos gases propano e butano, mas, não conseguimos encontrar um valor tabelado nas literaturas para a velocidade do som nem no GLP, nem no propano e nem mesmo no butano. Desse modo, tomaremos o valor experimental como a conclusão de um experimento, pois não temos dados o suficiente para conferir se nosso valor foi ou não próximo o suficiente da real velocidade do som no gás liquefeito do petróleo (GLP). 
