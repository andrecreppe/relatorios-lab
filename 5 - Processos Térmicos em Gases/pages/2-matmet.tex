\newpage
\section{Materiais e métodos}

% =============== INTRODUÇÃO ===================== %
\subsection{O fator $\gamma$ do ar}

O coeficiente de expansão adiabática, representado pela letra grega $\gamma$, é a razão entre a capacidade térmica a pressão constante e a capacidade térmica a volume constante de um gás, ou seja:

\[ \gamma = \frac{c_P}{c_V} \]

Nesse tipo de transformação - adiabática -, o sistema não troca calor com o ambiente; o trabalho realizado é referente à variação de energia interna do sistema. Numa expansão adiabática, o sistema realiza trabalho sobre o meio e a energia interna diminui. Na expansão adiabática ocorre um abaixamento de temperatura.\\

Pela Lei dos Gases Ideais, juntamente do Teorema da Equipartição de Energia e, também, por outras equações de termodinâmica, pode-se chegar às equações que relacionam $\gamma$ com as capacidades caloríficas a volume e pressão constantes ($C_V$ e $C_P$, respectivamente).\\

Desse modo, como o calor específico pode ser obtido pela divisão da capacidade calorífica pela massa, conseguimos obter a equação que relaciona $\gamma$ com $c_V$ e $c_P$ - já citada acima -, equação que será base para nossos primeiros experimentos.

% =============== EXPERIMENTOS ===================== %

\subsection{Fator $\gamma$ do ar: Método de Clément – Desormes}

aaaaaaaaaaaaaaaaaaaa

\subsection{Fator $\gamma$ do ar: Método de Rüchardt}

Primeiramente, o experimento foi montado pelos responsáveis do laboratório e disponibilizados por meio das gravações de vídeo. Assim sendo, nos pudemos extrair os seguintes dados e convertendo-os para o SI temos:

\begin{table}[H]
    \centering
    \begin{tabular}{ |c||c||c||c|  }
        \hline
        \textbf{O que foi medido} & \textbf{Valor} & \textbf{Incerteza} & \textbf{Unidade}\\
        \hline 
         Massa da bolinha (m)       & 0,01672   & 0,00001   & kg\\
         Diâmetro do tubo (d)       & 0,0160    & 0,0001    & m\\
         Volume do recipiente (V)   & 0,0104    & 0,0001    & $m^3$\\
         \hline
         Período de 4 oscilações ($T_4$) & 4,82 & 0,01 & s\\
         \hline
         Pressão atmosférica (P) & 92125,8 & 0,1 & Pa\\
        \hline
    \end{tabular}
    \caption{Dados físicos e dimensões do experimento de Rüchardt} 
\end{table}

Com os dados em mãos podemos calcular tudo que precisamos. Antes de tudo, entretanto, é necessário obter o período de apenas uma oscilação (T), o que é simples, pois basta dividir $T_4$ por 4:

\[ T = \frac{T_4}{4} = \frac{4,82}{4} = 1,205 s \]
\[ \delta T = \frac{\delta T_4}{4} = \frac{0.01}{4} = 0,0025 s \]

Além disso, precisamos da área do tubo onde a bolinha estava suspensa. Mais uma vez, esse é um resultado bem fácil de ser obtido, já que temos o diâmetro do tubo e a seção do tubo é um círculo:

\[ r = \frac{d}{2} = \frac{0,0160}{2} = 0,008 m \]
\[ A = \pi r^2 = 3,14159 \cdot (0,008)^2 = 0.0002010619 m^2 \]

\[ \delta r = \frac{\delta d}{2} = \frac{0,0001}{2} = 0,00005 m \]
\[ \delta A = \pi (\delta r)^2 = 3,14159 \cdot (0,00005)^2 = 0.00000000785 m^2 \]

Agora sim podemos calcular o nosso fator $\gamma$ do ar atmosférico, que é uma tarefa bem direta, pois basta inserirmos os dados experimentais na equação geral:

\[ N = m \cdot V \]
\[ N = 0,01672 \cdot 0,0104 = 0,000173888 \]

\[ D = P \cdot A^2 \cdot T^2 \]
\[ D = 92125,8 \cdot (0,0002010619)^2 \cdot (1,205)^2 = 0,0054077307 \]

\[ \gamma = 4\pi^2 \cdot \frac{N}{D} \]
\[ \gamma = 4\pi^2 \cdot \frac{0,000173888}{0,0054077307} = 1,26944617 \]

Agora, propagando as incertezas:



\subsection{Zero absoluto: Determinação do zero absoluto utilizando um termômetro a
gás}

A terceira prática tem como objetivo determinar a temperatura do zero absoluto. Para isso, será utilizado o termômetro a gás, a volume constante, que consiste em um bulbo de vidro contendo hélio, que é então ligado a um barômetro do tipo Torricelli. Esse sistema pode ser observado na representação esquemática a seguir:

\begin{figure}[H]
  \centering
  \includegraphics[scale=0.9]{images/Termômetro a Gás.png}
  \caption{Representação do Termômetro a Gás.}
\end{figure}

O termômetro é formado por um tubo em “U” contendo mercúrio em seu interior e com um dos braços lacrados para que a pressão em seu interior seja zero. No outro braço é colocado um balão de vidro contendo gás hélio a uma pressão próxima da pressão atmosférica. Para a leitura da pressão nesse barômetro, basta observar que a pressão exercida pelo gás Hélio em um ponto A do braço é exatamente igual à pressão exercida pela coluna de mercúrio sobre o ponto B do outro braço, a qual pode ser obtida diretamente pela sua altura h em cmHg (centímetro de mercúrio).

\begin{figure}[H]
  \centering
  \includegraphics[scale=0.75]{images/Termômetro a Gás esquemático.png}
  \caption{Esquemático dos elementos que compõem o termômetro a gás.}
\end{figure}

O experimento consiste, então, em medir a pressão do gás para diversas temperaturas obtidas na seguinte ordem:

\begin{enumerate}
    \item Bulbo mergulhado em água à temperatura ambiente;
    \item Bulbo mergulhado em gelo em fusão;
    \item Bulbo mergulhado em nitrogênio líquido;
    \item Bulbo mergulhado em água em ebulição.
\end{enumerate}

Após isso, será construído um gráfico da pressão (medida em cmHg) em função da temperatura (medida em ºC). A partir do método dos mínimos quadrados, será determinado o coeficiente de dilatação do gases ideais a volume constante ($\beta$) e o valor de $P_0$. Essa relação pode ser comprovada a partir de uma equação em que ao aumentar a temperatura de um gás, mantido volume constante, a pressão varia linearmente com a temperatura. Se a temperatura inicial do gás é 0ºC e sua pressão inicial é $P_0$, a pressão P(T), à temperatura T(ºC), será dada por:

\[ P(T) = P_0 \cdot (\beta \cdot T + 1) = P_0 \cdot \beta \cdot T + P_0 \]\

Nesse caso, $P_0$ corresponde ao coeficiente linear da reta e o valor do coeficiente $\beta$ pode ser relacionado diretamente ao coeficiente angular, que nesse caso será igual a $\beta P_0$.\\

Com os valores de $\beta$ e $P_0$, escreve-se a equação que descreve o comportamento da variação da pressão do gás em função da temperatura. Utilizando essa equação, será construído uma reta sobre os pontos experimentais e a partir da extrapolação desta reta será determinado a temperatura zero absoluto. Essa situação corresponde à menor temperatura que se pode alcançar fisicamente.\\
